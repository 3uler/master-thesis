\documentclass[12pt,a4paper,twoside]{book}
\usepackage[utf8]{inputenc}
\usepackage[german,english]{babel} 
\usepackage{amsmath}
\usepackage{amsfonts}
\usepackage{amssymb}
\usepackage{graphicx}
%\usepackage{subfig}
\usepackage[left=4cm,right=3.5cm,top=2.5cm,bottom=3.5cm]{geometry}
%\usepackage[left=3cm,right=3cm,top=2.5cm,bottom=3.cm]{geometry}
\usepackage{setspace}
\usepackage{babelbib}
\usepackage{bm}
\usepackage{upgreek}
\usepackage[nottoc]{tocbibind}
\usepackage{tikz}
\usepackage{dsfont}
%\usepackage{mathabx}
\usepackage[toc,page]{appendix}
%erweiterungen
\usepackage{titlesec,titletoc}	
%\usepackage{scrtime}
\usepackage{textcomp}
\usepackage{upgreek}
\usepackage{framed, color} 
\usepackage[colorlinks=false,hidelinks]{hyperref}
\usepackage{tcolorbox}
\usepackage{anyfontsize}
\usepackage{booktabs}
\usepackage{ifthen}
\usepackage{longtable}
\usepackage{sansmath}

\usepackage[labeled,resetlabels]{multibib}	%%meherere literaturverzeichnisse

% declare new math alphabets for special bolt and italic fonts
\DeclareMathAlphabet{\mathcalbf}{OMS}{cmsy}{b}{n}
\DeclareMathAlphabet{\mathitbf}{OT1}{cmss}{bx}{it}

%change name displayed as appendix title
\renewcommand{\appendixpagename}{\sffamily \bfseries \huge Appendix}

\graphicspath{{/Users/Philipp/Documents/Uni/Masterarbeit/latex/draft/graphics/}}








%layout einstellungen

%equation nummerierungsformat
\renewcommand{\theequation}{\arabic{section}.\arabic{equation}} 
\numberwithin{equation}{section}


%caption-formatierung

\numberwithin{figure}{section}	%abbildungszähler
\numberwithin{table}{section}	%tabellenzähler

%sidecap formatierung (ergänzung zur figure und table umgebung)
\usepackage[outercaption]{sidecap}

\usepackage[margin=0pt,font=small,labelformat=simple,labelfont=bf,labelsep=space, justification=justified, hypcap]{caption}
%\captionsetup[figure]{name=Figure}
%\captionsetup[table]{name=Table}
%\captionsetup[table]{position=above}
%zur positionuierung einer SC caption
\makeatletter
\newenvironment{SCtopfig}{\SC@float[t]{figure}}{\endSC@float}
\makeatother






%header einstellungen
\usepackage{fancyhdr}
\pagestyle{fancy}
\fancyhead{}
\fancyhead[RO]{ \nouppercase{\rightmark}}
\fancyhead[LE]{ \nouppercase{\leftmark}} %\slshape

\fancyfoot{}
\fancyfoot[CE,CO]{\thepage}




%Einstellung serifenlose überschriften
\titleformat{\chapter}[hang]{\sffamily \bfseries \Large}{\framebox[0.9cm]{\mdseries\fontsize{40}{0}\selectfont\thechapter}}{ 0.5cm}{ }[\vspace{-0.65cm}\rule{\textwidth}{1pt}]
\titleformat*{\section}{\sffamily \bfseries \Large}
\titleformat*{\subsection}{\sffamily \bfseries \large}
\titleformat*{\subsubsection}{\sffamily \bfseries \normalsize}









%%EIGENDEFINITIONEN

\definecolor{grey}{rgb}{0.925,0.925,0.925}
\newcommand{\ueberschrift}[1]{\vspace{2mm}{\sffamily\normalsize\bfseries {#1} \vspace{2mm}}}
\newcommand{\ergebnis}[1]{\vspace{2mm}
\fcolorbox{black}{grey}{\parbox{\columnwidth}{
\begin{equation}
{#1}
\end{equation}
}}\vspace{2mm}}
\newcommand{\abs}[1]{\ensuremath{\left\vert#1\right\vert}}
%differential d
\newcommand{\dd}{\ensuremath{\text{d}}}
%differential del
\newcommand{\del}{\ensuremath{\partial}}
%index real part
\newcommand{\R}{\ensuremath{^{\text{R}}}}
\newcommand{\dR}{\ensuremath{_{\text{R}}}}
%index imag part
\newcommand{\I}{\ensuremath{^{\text{I}}}}
\newcommand{\dI}{\ensuremath{_{\text{I}}}}

% AdS_5 x S^5
\newcommand{\AdSS}{\ensuremath{AdS_{5}\times S^{5}}}



% definition of names to be written in a specific font
\newcommand{\names}[1]{\textsc{#1}}

% Pfaffian
\newcommand{\pf}{\ensuremath{\text{Pf}\,}}

%backward vector arrow
\makeatletter
\DeclareRobustCommand{\cev}[1]{%
  \mathpalette\do@cev{#1}%
}
\newcommand{\do@cev}[2]{%
  \fix@cev{#1}{+}%
  \reflectbox{$\m@th#1\vec{\reflectbox{$\fix@cev{#1}{-}\m@th#1#2\fix@cev{#1}{+}$}}$}%
  \fix@cev{#1}{-}%
}
\newcommand{\fix@cev}[2]{%
  \ifx#1\displaystyle
    \mkern#23mu
  \else
    \ifx#1\textstyle
      \mkern#23mu
    \else
      \ifx#1\scriptstyle
        \mkern#22mu
      \else
        \mkern#22mu
      \fi
    \fi
  \fi
}
\makeatother



\title{Geodäten im Gravitationsfeld geladener Staubwolken}
\author{Philipp Töpfer}