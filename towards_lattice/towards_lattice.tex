\section{Towards the lattice simulation}
With the current status of the action (\ref{eq: cusp_action}) we could almost start to discretize the operators and fields, at least for the bosonic part this would not be a problem. For the fermions however this is not so straight forward. In order to include the fermionic contribution into the weight factor of the path integral like explained in section \ref{sec: hmc_alg} one needs to integrate out the \names{Grassmann} variables to result into a \names{Pfaffian} or determinant of a fermionic operator. As presented in Appendix \ref{sec: grassmann_analysis} this is only possible if the fermions appearing are of quadratic order. But in the fluctuation action (\ref{eq: cusp_action}) also quartic contributions of fermions appear, which have to be linearized with help of a \names{Hubbard-Statonovich} transformation.\footnote{See Appendix \ref{sec: grassmann_analysis} for details}
%
%
%
% - - - - - - - - -  linearization of the action  - - - - - - - - - - - -
%
%
%
\subsection{Linearization of fermionic contributions}
\subsubsection{Naive approach and sign problem}
The only quartic interactions are coming from the $\eta$ fields and we can write this part of the action as
%
%
\begin{equation}
S^{\rm F}_{4} [\eta_{i},\eta^{i}] = g \int \dd t \dd s \, \left[-\frac{1}{z^{2}}\left(\eta^{2}\right)^{2} + \left(\frac{i}{z^{2}}z_{N}
\eta_{i}{\left(\rho^{MN}\right)^{i}}_{j}\eta^{j}\right)^{2} \right].
\label{eqq: S_4}
\end{equation}
%
%
In the path integral representation the euclidean action contributes within an exponential $e^{-S_{\rm E}}$. By performing a naive \names{Hubbard-Statonovich} transformation to this exponential we can reduce the four-fermion contributions to quadratic \names{Yukawa} terms whereas we have to introduce 7 bosonic real auxiliary fields $\phi$ and $\phi^{N}$
%
%
\begin{align}
\exp \bigg\lbrace -g \int \dd t \dd s \, \bigg[-\tfrac{1}{z^{2}} \big(\eta^{2}\big)^{2} + \Big(\tfrac{i}{z^{2}}z_{N}
& \eta_{i}{\big(\rho^{MN}\big)^{i}}_{j}\eta^{j}\Big)^{2} \bigg] \bigg\rbrace \\
%
%
\sim \int \mathcal{D}\phi \mathcal{D}\mathcal{\phi^{M}}\; \exp\bigg\lbrace -g \int \dd t \dd s \, \bigg[ &\tfrac{1}{2}\phi^{2} + \tfrac{\sqrt{2}}{z}\phi\eta^{2} +\tfrac{1}{2} (\phi^{M})^{2} \notag \\
 &- i\tfrac{\sqrt{2}}{z^{4}} \phi_{M} \left( i\eta_{i}{(\rho^{MN})^{i}}_{j}\eta^{j}\right) z_{N} \bigg] \bigg\rbrace. \notag
\end{align}
%
%
Here we can notice that the second term appears to be complex, since the $SO(6)$ matrix in parenthesis is hermitian (with respect to the indices $M,N$)
%
%
\begin{equation}
\left( i \,\eta_{i} {\left(\rho^{MN}\right)^{i}}_{j}\eta^{j}\right)^{\dagger} = i\, \eta_{j} {\left(\rho^{MN}\right)^{j}}_{i}\eta^{i},
\end{equation}
%
%
where we have used (\ref{eq: SO6_id}). As discussed in section \ref{sec: sign_prob} this complex phase in the weight function is potentially leading to a non treatable sign problem. We therefore chose to make a field redefinition that circumvents the appearance of a complex phase during the HS transformation. 
%
%
%
% - - - - alternative field def - - - - - 
%
%
%
\subsubsection{Alternative field redefinition}
By using the identities for the $SO(6)$ matrices stated in Appendix \ref{sec: SO6} we can rewrite the second term in the \names{Laggrangian} of (\ref{eqq: S_4}) as
%
%
\begin{equation}
\left(i\, \eta_i {(\rho^{MN})^i}_j n^N \eta^j\right)^2=-3 (\eta^2)^2+2\eta_i (\rho^N)^{ik} n_N \eta_k \eta^j (\rho^L)_{jl} n_L \eta^l,
\end{equation}
%
%
where we defined $n^{N}=\tfrac{z^{N}}{z}$. This leads to the \names{Lagrangian}
%
%
\begin{equation}
\mathcal{L}_4=\frac{1}{z^2}\left(- 4\, (\eta^2)^2+2\left|\eta_i (\rho^N)^{ik} n_N \eta_k\right|^2\right).
\end{equation}
%
%
In order to circumvent the sign problem the second term needs to be negative. To achieve this we define new fields\footnote{Where we actually set ${\Sigma_i}^j=\eta_i \eta^j$, then defined ${\Sigma^{i}}_{j}\equiv({\Sigma_i}^{j})^{*}={\Sigma_{j}}^{i}$ to emphasize the notation $\Sigma_i^j$ and equivalent for $\tilde{\Sigma}$.}
%
%
\begin{equation}
\Sigma_i^j=\eta_i \eta^j \qquad \tilde{\Sigma}_j^i=(\rho^N)^{ik}n_N (\rho^L)_{jl}n_L \eta_k \eta^l.
\end{equation}
%
%
with this new definitions it is simple to check that
%
%
\begin{align}
 \Sigma^j_{i}\Sigma^i_j&=-(\eta^2)^2 & \tilde\Sigma^j_{i} \tilde\Sigma^i_j&=-(\eta^2)^2 & \Sigma^i_j \tilde\Sigma^j_i&=-\left|\eta_i (\rho^N)^{ik} n_N \eta_k\right|^2.
\end{align}
%
%
With this we now define
%
%
\begin{equation}
{\Sigma_{\pm}}_i^j=\Sigma_i^j\pm \tilde \Sigma_i^j
\end{equation}
%
%
and find
%
%
\begin{equation}
 {\Sigma_{\pm}}_i^j{\Sigma_{\pm}}_j^i=-2(\eta^2)^2 \mp 2\left|\eta_i (\rho^N)^{ik} n_N \eta_k\right|^2.
\end{equation}
%
%
We can now substitute the new fields into the \names{Lagrangian} and obtain
%
%
\begin{equation}
\mathcal{L}_4=\frac{1}{z^2}\left(- 4\, (\eta^2)^2 \mp 2(\eta^2)^2 \mp {\Sigma_{\pm}}_i^j{\Sigma_{\pm}}_j^i \right)\,,
\end{equation}
%
%
where we only need to select the right sign in the field definition to overcome the sign problem, which is leading to
%
%
\begin{equation}
\mathcal{L}_4= \frac{1}{z^2}\left(- 6\, (\eta^2)^2 - {\Sigma_{+}}_i^j{\Sigma_{+}}_j^i \right)\,.
\end{equation}
%
%
If we now perform a HS transformation there will be no complex phase. The HS transformation yields
%
%
\begin{equation}
 -\frac{6}{z^2}(\eta^2)^2\to \frac{12}{z} \eta^2 \phi +6\phi^2,
\end{equation}
%
%
where a single bosonic field was introduced like in the naive case. And further
%
%
\begin{equation}
-\frac{1}{z^{2}}{\Sigma_{+}}^{i}_{j}{\Sigma_{+}}^{j}_{i} \rightarrow \frac{2}{z}{\Sigma_{+}}^{i}_{j}\phi^{j}_{i} + \phi^{i}_{j}\phi^{j}_{i}
\qquad \text{with} \qquad \left(\phi^{i}_{j}\right)^{\ast} = \phi^{j}_{i}\;,
\end{equation}
%
%
Here the collection of fields $\phi^{i}_{j}$ can be thought of as a complex hermitian matrix with 16 real free parameters. After reinserting the old fields for $\Sigma_{+}$ we can conclude