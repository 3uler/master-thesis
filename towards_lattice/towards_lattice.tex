\chapter{Towards the lattice simulation}
\label{sec: towards_lat}
With the current status of the action (\ref{eq: cusp_action}) we could almost start to discretize the operators and fields, at least for the bosonic part this would not be a problem. For the fermions however this is not so straight forward. In order to include the fermionic contribution into the weight factor of the path integral like explained in section \ref{sec: hmc_alg} one needs to integrate out the \names{Grassmann} variables to result into a \names{Pfaffian} or determinant of a fermionic operator. As presented in Appendix \ref{sec: grassmann_analysis} this is only possible if the fermions appearing are of quadratic order. But in the fluctuation action (\ref{eq: cusp_action}) also quartic contributions of fermions emerge, which have to be linearized first with help of a \names{Hubbard-Stratonovich} transformation.\footnote{See Appendix \ref{sec: grassmann_analysis} for details}
%
%
%
% - - - - - - - - -  linearization of the action  - - - - - - - - - - - -
%
%
%
\section{Linearization of fermionic contributions}
\subsection{Naive approach and sign problem}
The only quartic interactions are coming from the $\eta$ fields and we can write this part of the action as
%
%
\begin{equation}
S^{\rm F}_{4} [\eta_{i},\eta^{i}] = g \int \dd t \dd s \, \left[-\frac{1}{z^{2}}\left(\eta^{2}\right)^{2} + \left(\frac{i}{z^{2}}z_{N}
\eta_{i}{\left(\rho^{MN}\right)^{i}}_{j}\eta^{j}\right)^{2} \right].
\label{eqq: S_4}
\end{equation}
%
%
In the path integral representation the \names{Euclidean} action contributes within an exponential $e^{-S_{\rm E}}$. By performing a naive \names{Hubbard-Statonovich} transformation on this exponential we can reduce the four-fermion contributions to quadratic \names{Yukawa} terms whereas we have to introduce 7 bosonic real auxiliary fields $\phi$ and $\phi^{N}$
%
%
\begin{align}
\exp \bigg\lbrace -g \int \dd t \dd s \, \bigg[-\tfrac{1}{z^{2}} \big(\eta^{2}\big)^{2} + \Big(\tfrac{i}{z^{2}}z_{N}
& \eta_{i}{\big(\rho^{MN}\big)^{i}}_{j}\eta^{j}\Big)^{2} \bigg] \bigg\rbrace \\
%
%
\sim \int \mathcal{D}\phi \mathcal{D}\mathcal{\phi^{M}}\; \exp\bigg\lbrace -g \int \dd t \dd s \, \bigg[ &\tfrac{1}{2}\phi^{2} + \tfrac{\sqrt{2}}{z}\phi\eta^{2} +\tfrac{1}{2} (\phi^{M})^{2} \notag \\
 &- i\tfrac{\sqrt{2}}{z^{4}} \phi_{M} \left( i\eta_{i}{(\rho^{MN})^{i}}_{j}\eta^{j}\right) z_{N} \bigg] \bigg\rbrace. \notag
\end{align}
%
%
Here we can notice that the second term appears to be complex, since the $SO(6)$ matrix in parenthesis is hermitian (with respect to the indices $M,N$)
%
%
\begin{equation}
\left( i \,\eta_{i} {\left(\rho^{MN}\right)^{i}}_{j}\eta^{j}\right)^{\dagger} = i\, \eta_{j} {\left(\rho^{MN}\right)^{j}}_{i}\eta^{i},
\end{equation}
%
%
where we have used (\ref{eq: SO6_id}). As discussed in section \ref{sec: sign_prob} this complex phase in the weight function is potentially leading to a non treatable sign problem. We therefore chose to make a field redefinition that circumvents the appearance of a complex phase during the HS transformation.
%
%
%
% - - - - alternative field def - - - - -
%
%
%
\subsection{Alternative field redefinition}
By using the identities for the $SO(6)$ matrices stated in Appendix \ref{sec: SO6} we can rewrite the second term in the \names{Lagrangian} of (\ref{eqq: S_4}) as
%
%
\begin{equation}
\left(i\, \eta_i {(\rho^{MN})^i}_j n^N \eta^j\right)^2=-3 (\eta^2)^2+2\eta_i (\rho^N)^{ik} n_N \eta_k \eta^j (\rho^L)_{jl} n_L \eta^l,
\end{equation}
%
%
where we defined $n^{N}=\tfrac{z^{N}}{z}$. This leads to the \names{Lagrangian}
%
%
\begin{equation}
\mathcal{L}_4=\frac{1}{z^2}\left(- 4\, (\eta^2)^2+2\left|\eta_i (\rho^N)^{ik} n_N \eta_k\right|^2\right).
\end{equation}
%
%
In order to circumvent the sign problem the second term needs to be negative. To achieve this we define new fields\footnote{Where we actually set ${\Sigma_i}^j=\eta_i \eta^j$, then defined ${\Sigma^{i}}_{j}\equiv({\Sigma_i}^{j})^{*}={\Sigma_{j}}^{i}$ to emphasize the notation $\Sigma_i^j$ and equivalent for $\tilde{\Sigma}$.}
%
%
\begin{equation}
\Sigma_i^j=\eta_i \eta^j \qquad \tilde{\Sigma}_j^i=(\rho^N)^{ik}n_N (\rho^L)_{jl}n_L \eta_k \eta^l.
\end{equation}
%
%
with this new definitions it is simple to check that
%
%
\begin{align}
 \Sigma^j_{i}\Sigma^i_j&=-(\eta^2)^2 & \tilde\Sigma^j_{i} \tilde\Sigma^i_j&=-(\eta^2)^2 & \Sigma^i_j \tilde\Sigma^j_i&=-\left|\eta_i (\rho^N)^{ik} n_N \eta_k\right|^2.
\end{align}
%
%
With this we then define
%
%
\begin{equation}
{\Sigma_{\pm}}_i^j=\Sigma_i^j\pm \tilde \Sigma_i^j
\end{equation}
%
%
and find
%
%
\begin{equation}
 {\Sigma_{\pm}}_i^j{\Sigma_{\pm}}_j^i=-2(\eta^2)^2 \mp 2\left|\eta_i (\rho^N)^{ik} n_N \eta_k\right|^2.
\end{equation}
%
%
We can thus substitute the new fields into the \names{Lagrangian} and obtain
%
%
\begin{equation}
\mathcal{L}_4=\frac{1}{z^2}\left(- 4\, (\eta^2)^2 \mp 2(\eta^2)^2 \mp {\Sigma_{\pm}}_i^j{\Sigma_{\pm}}_j^i \right)\,,
\end{equation}
%
%
where we only need to select the right sign in the field definition to overcome the sign problem, which is leading to
%
%
\begin{equation}
\mathcal{L}_4= \frac{1}{z^2}\left(- 6\, (\eta^2)^2 - {\Sigma_{+}}_i^j{\Sigma_{+}}_j^i \right)\,.
\end{equation}
%
%
If we then perform a HS transformation there will be no complex phase. The HS transformation yields
%
%
\begin{equation}
 -\frac{6}{z^2}(\eta^2)^2\to \frac{12}{z} \eta^2 \phi +6\phi^2,
 \label{eq: HS_phi}
\end{equation}
%
%
where a single bosonic field was introduced like in the naive case. And further
%
%
\begin{equation}
-\frac{1}{z^{2}}{\Sigma_{+}}^{i}_{j}{\Sigma_{+}}^{j}_{i} \rightarrow \frac{2}{z}{\Sigma_{+}}^{i}_{j}\phi^{j}_{i} + \phi^{i}_{j}\phi^{j}_{i}
\qquad \text{with} \qquad \left(\phi^{i}_{j}\right)^{\ast} = \phi^{j}_{i}\;.
\end{equation}
%
%
Here the collection of fields $\phi^{i}_{j}$ can be thought of as a complex hermitian matrix with 16 real free parameters. We find it convenient to rescale the field $\phi \to \phi / \sqrt{6}$, to get rid of the pre factor of 6 in (\ref{eq: HS_phi}). After reinserting the old fields for $\Sigma_{+}$ we can conclude that
%
%
\begin{equation}
 \mathcal{L}_4\to \frac{12}{z} \eta^2 \phi +\phi^2+\frac{2}{z}\eta_j \phi^j_i \eta^i +\frac{2}{z} (\rho^N)^{ik}n_N \eta_k\phi^j_i  (\rho^L)_{jl}n_L  \eta^l+\phi^i_j \phi^j_i\,,
\end{equation}
%
%
and we can write the full \names{Lagrangian} as
%
%
\begin{align}
\mathcal{L}_{\rm cusp} &=  {\left\vert \partial_t {x} + {\frac{m}{2}}{x} \right\vert}^2 + \frac{1}{{ z}^4}{\left\vert \partial_s {x} -\frac{m}{2}{x} \right\vert}^2 + \left(\partial_t {z}^M + \frac{m}{2}{z}^M \right)^2 \\ &\quad+ \frac{1}{{ z}^4} \left(\partial_s {z}^M -\frac{m}{2}{z}^M\right)^2
+ \phi^2 + {\rm tr} \left( \tilde{\phi}\, \tilde{\phi}^{\dagger} \right) + \mathit{\Psi}^T \mathcal{O}_{\rm F} \mathit{\Psi} \,. \notag
\end{align}
%
%
Hereby we defined the fermionic vector $\mathit{\Psi}\equiv (\theta^{i},\theta_{i},\eta^{i},\eta_{i})$ as well as the auxiliary matrix ${\tilde{\phi}= (\tilde{\phi}_{ij}) \equiv \phi^{i}_{j}}$. We used partial integration and the properties of the \names{Grassmann} numbers and $SO(6)$ matrices to write the fermionic contribution in a matrix-vector notation. The $16\times 16$ fermionic operator is hereby represented as $4\times 4$ block matrix
%
%
\begingroup
\everymath{\footnotesize}
%\normalsize
\begin{flalign} \label{OF}
\!\!\!\!\!\!\!\!
\mathcal{O}_{\rm F} & =\begin{pmatrix}
0 & i \mathds{1}_{4}\partial_{t} & -i\rho^{M}\left(\partial_{s}+\frac{m}{2}\right)\frac{{z}^{M}}{{z}^{3}} & 0\\
i \mathds{1}_{4}\partial_{t} & 0 & 0 & -i\rho_{M}^{\dagger}\left(\partial_{s}+\frac{m}{2}\right)\frac{{z}^{M}}{{z}^{3}}\\
i\frac{{z}^{M}}{{z}^{3}}\rho^{M}\left(\partial_{s}-\frac{m}{2}\right) & 0 & 2\frac{{z}^{M}}{{z}^{4}}\rho^{M}\left(\partial_{s}{x}-m\frac{{x}}{2}\right) & i \mathds{1}_{4}\partial_{t}-A^{T}\\
0 & i\frac{{z}^{M}}{{z}^{3}}\rho_{M}^{\dagger}\left(\partial_{s}-\frac{m}{2}\right) & i \mathds{1}_{4}\partial_{t}+A & -2\frac{{z}^{M}}{{z}^{4}}\rho_{M}^{\dagger}\left(\partial_{s}{x}^\ast-m\frac{{x}}{2}^\ast\right)
\end{pmatrix}~,
\raisetag{-8pt}
\end{flalign}
\endgroup
%
%
where
%
%
\begin{equation}
A=-\frac{\sqrt{6}}{z}\phi \mathds{1}_{4} + \frac{1}{z}\tilde{\phi}+\frac{1}{z^{3}}\rho^\ast_{N}\tilde{\phi}^{T}\rho^{L}z^{N}z^{L}+\mathrm{i}\frac{z^{N}}{z^2}\rho^{MN}\partial_{t}z^{M}.
\end{equation}
%
%
The auxiliary matrix $\tilde{\phi}$ is constructed from 16 real auxiliary fields $\phi_{I}\; (I=1,\ldots,16)$ in the following way
%
%
\begin{equation}
\tilde{\phi} = \frac{1}{\sqrt{2}}
\begin{pmatrix}
\sqrt{2}\phi_{13} & \phi_{1}+i\phi_{2} & \phi_{3}+i\phi_{4} & \phi_{5}+i\phi_{6} \\
\phi_{1}-i\phi_{2} & \sqrt{2}\phi_{14} & \phi_{7}+i\phi_{8} & \phi_{9}+i\phi_{10} \\
\phi_{3}-i\phi_{4} & \phi_{7}-i\phi_{8} & \sqrt{2}\phi_{15} & \phi_{11}+i\phi_{12} \\
\phi_{5}-i\phi_{6} & \phi_{9}-i\phi_{10} & \phi_{11}-i\phi_{12} & \sqrt{2}\phi_{16}
\end{pmatrix} ,
\end{equation}
%
%
so that we have the simple expression
%
%
\begin{equation}
{\rm tr}\left(\tilde{\phi}\, \tilde{\phi}^{\dagger}\right) = \sum\limits_{I=1}^{16} (\phi_{I})^{2} \equiv (\phi_{I})^{2}\,.
\end{equation}
%
%
%
%
%
% - - - - - - - -  matrix properties  - - - - - - - - -
%
%
%
%
%
%
\subsection{Matrix properties} \label{sec: matrix_prop}
The fermion matrix obeys some fundamental symmetries that shall be summarized in this section. Since $\mathcal{O}_{\rm F}$ is acting as a bilinear together with the anticommuting \names{Grassmann} fields, it is clear that the matrix representation of $\mathcal{O}_{\rm F}$ needs to be skew-symmetric
%
%
\begin{equation}
\mathcal{O}_{\rm F}^{\rm T} = -\mathcal{O}_{\rm F}.
\label{eq: skew_sym}
\end{equation}
%
%
It further possesses a global $U(1)$ and $SO(6)$ symmetry. The $U(1)$ symmetry manifests itself through the fact that certain blocks of $\mathcal{O}_{\rm F}$ are zero. As we saw from (\ref{eq: U1_sym}) fermions transform under $U(1)$ according to a certain charge $q$
%
%
\begin{equation}
\psi \to e^{iq\alpha}\psi.
\end{equation}
%
%
Therefore only fermions with complementary charges are allowed to couple in order to respect the $U(1)$ symmetry, and the blocks that lead to other couplings need to be zero. The $SO(6)$ symmetry requires the $4\times 4$ block structure and that each block is built from the $SO(6)$ invariant structures: $\mathds{1}_{4},\,\rho^{M}u^{M},\,\rho_{M}^{\dagger}u^{M}$. The fermion matrix obeys also another constraint which is reminiscent of the $\gamma_{5}-$hermiticity in lattice QCD \cite{montvay_lattice}
%
%
\begin{equation}
\mathcal{O}_{\rm F}^{\dagger} = \Gamma_{5}\mathcal{O}_{\rm F}\Gamma_{5},
\label{eq: G5_sym}
\end{equation}
%
%
where $\Gamma_{5}$ is the following unitary, antihermitian matrix
%
%
\begin{equation}
\Gamma_{5} = \begin{pmatrix}
0 & \mathds{1}_{4} & 0 & 0 \\
 -\mathds{1}_{4}& 0 & 0 & 0 \\
0 & 0 & 0 & \mathds{1}_{4} \\
0 & 0 & -\mathds{1}_{4} & 0
\end{pmatrix} ,\qquad
\Gamma_{5}^{\dagger}\Gamma_{5}= \mathds{1}_{16}, \quad \Gamma_{5}^{\dagger} = -\Gamma_{5}.
\end{equation}
%
%
A general skew-symmetric block matrix $M$ that is $\Gamma_{5}$-hermitian needs to have the structure
%
%
\begin{equation}
M=\begin{pmatrix}
d_{1} & a & b & c \\
-a^{\rm T} & -d_{1}^{\dagger} & -c^{*} & b^{*} \\
-b^{\rm T} & c^{\dagger} & d_{2} & f \\
-c^{\rm T} & -b^{\dagger} & -f^{\rm T} & -d_{2}^{\dagger}
\end{pmatrix} ,\qquad d_{i}=-d_{i}^{\rm T},\quad a=a^{\dagger}, \quad f=f^{\dagger}.
\label{eq: G5_herm_mat}
\end{equation}
%
%
We can use the two properties (\ref{eq: skew_sym}) and (\ref{eq: G5_sym}) to ensure the absence of a complex phase in the determinant
%
%
\begin{align}
\det \left(\mathcal{O}_{\rm F}\right)^{*} &= \det  \left( \mathcal{O}_{\rm F}^{\dagger}\right) = \det \left( \Gamma_{5}\mathcal{O}_{\rm F} \Gamma_{5} \right) \notag \\
%
%
&= \det \left(\Gamma_{5}\right)^{2} \det \left(\mathcal{O}_{\rm F}\right) \\
%
%
&= \det \left(\mathcal{O}_{\rm F}\right) \notag
\end{align}
%
%
and thereby follows that $\det\mathcal{O}_{\rm F} \in \mathbb{R}$. For the \names{Pfaffian} of $\mathcal{O}_{\rm F}$ to be nonnegative we require that $\det \mathcal{O}_{\rm F}$ is positive and factorizable into two equivalent terms, so that
%
%
\begin{equation}
\det \mathcal{O}_{\rm F} = \text{Pf}\,(\mathcal{O}_{\rm F})^{2}.
\end{equation}
%
%
Therefore we want to examine the spectrum of $\mathcal{O}_{\rm F}$. Assuming that $\lambda$ is an eigenvalue of $\mathcal{O}_{\rm F}$ and $P(\lambda)$ is the characteristic polynomial we can prove
%
%
\begin{align}
P(\lambda) &= \det (\mathcal{O}_{\rm F} -\lambda\mathds{1}) = \det (\Gamma_{5}(\mathcal{O}_{\rm F} -\lambda\mathds{1})\Gamma_{5}) \notag \\
%
&= \det (\mathcal{O}_{\rm F}^{\dagger}+\lambda\mathds{1}) = \det (\mathcal{O}_{\rm F} +\lambda^{*}\mathds{1})^{*} \\
&= P(-\lambda^{*})^{*}, \notag
\end{align}
%
%
and thereby see that if $\lambda$ is an eigenvalue, then also $-\lambda^{*}$ is an eigenvalue. Since $\mathcal{O}_{\rm F}$ is skew-symmetric also $-\lambda$ and $\lambda^{*}$ must be eigenvalues. So if for a discretized version $\hat{\mathcal{O}}_{\rm F}$ all these eigenvalues would be complex with non-vanishing real and imaginary part we could write the determinant as
%
%
\begin{equation}
\det \hat{\mathcal{O}}_{\rm F} = \prod\limits_{i=1}^{N} \vert \lambda_{i}\vert^{2} \vert \lambda_{i} \vert^{2}
\end{equation}
%
%
and therefore
%
%
\begin{equation}
\text{Pf}\,\hat{\mathcal{O}}_{\rm F} =  \pm\prod\limits_{i=1}^{N} \vert \lambda_{i}\vert^{2},
\end{equation}
%
%
where $N=\vert \mathit{\Lambda}\vert /4$. Since the eigenvalues should behave like continuous functions along a HMC trajectory it should not be possible for the \names{Pfaffian} to change its sign throughout a trajectory. So if one chooses a starting configuration with positive \names{Pfaffian} it should remain nonnegative during the whole simulation and the \names{Pfaffian} should be a valid probability distribution. Yet it is not quite clear what happens if there are also purely real or imaginary eigenvalues appearing. If they come in the same kind of quartets then there should not be a problem, otherwise the behaviour is not fully understood yet. We therefore need to keep track of of the \names{Pfaffian} during simulation to legitimate its status as valid probability density.
%
%
%
%
%
% - - - - - - - - pseudofermions - - - - - - - - - - -
%
%
%
%
%
\subsection{Pseudofermionic weight function}
Now since we have linearised fermionic contributions to quadratic order, we are able to integrate over the \names{Grassmann} fields in the partition function
%
%
\begin{equation}
Z = \int \mathcal{D}x\,\mathcal{D}x^{*}\,\mathcal{D}z^{N}\,\mathcal{D}\phi\,\mathcal{D}\phi_{I}\,\mathcal{D}\mathit{\Psi}\; e^{-S}\,,
\label{eq: Z_string2}
\end{equation}
%
%
where we will split $S = S_{\rm B} + S_{\rm F}$ into its bosonic ($S_{\rm B}$) and fermionic ($S_{\rm F}$) contributions with
%
%
\begin{align}
S_{\rm B} &= g\,\int \dd t\dd s\; {\left\vert \partial_t {x} + {\frac{m}{2}}{x} \right\vert}^2 + \frac{1}{{ z}^4}{\left\vert \partial_s {x} -\frac{m}{2}{x} \right\vert}^2 + \left(\partial_t {z}^M + \frac{m}{2}{z}^M \right)^2 \notag \\
&\qquad\qquad+ \frac{1}{{ z}^4} \left(\partial_s {z}^M -\frac{m}{2}{z}^M\right)^2 + \phi^2 + (\phi_{I})^{2}\,, \label{eq: bos_act} \\[0.5cm]
S_{\rm F} &=  g\,\int \dd t\dd s\; \mathit{\Psi}^T \mathcal{O}_{\rm F} \mathit{\Psi}\,. \notag
\end{align}
%
%
As motivated in section \ref{sec: grassmann_analysis} the \names{Grassmann} integral over $\mathit{\Psi}$ will result in a \names{Pfaffian} ${\rm Pf}\,\mathcal{O}_{\rm F}$. To include the \names{Pfaffian} into the weight function we have to rewrite it in terms of pseudofermions $\xi$ as emphasized in section \ref{sec: hmc_alg}. To legitimately apply this procedure we need the \names{Pfaffian} to be real and non-negative. By our alternative approach to linearization we made sure to exclude any potential phase ambiguities. As we could see from the previous section we are convinced that the \names{Pfaffian} can be treated as a positive real quantity which we will assume to be true in the following. For this reason, we proceed as in \cite{Roiban} and introduce pseudofermions $\xi$ via
%
%
\begin{equation}
\int \mathcal{D}\mathit{\Psi}\; e^{-S_{\rm F}} = {\rm Pf}\,\mathcal{O}_{\rm F} =\left( \det \mathcal{O}_{\rm F}\,\mathcal{O}_{\rm F}^{\dagger} \right)^{\frac{1}{4}} = \int \mathcal{D}\xi\,\mathcal{D}\xi^{\dagger}\; e^{-S_{\xi}}\,,
\end{equation}
%
%
where
%
%
\begin{equation}
S_{\xi} = g\,\int \dd t\dd s \; \xi^{\dagger} \left(\mathcal{O}_{\rm F}\,\mathcal{O}_{\rm F}^{\dagger} \right)^{-\frac{1}{4}} \xi \,,
\label{eq: S_xi}
\end{equation}
and $\xi$ is a collection of 16 complex bosonic scalar fields. But before we can go any further we need to discretize the action with help of the methods introduced in \autoref{sec: disc_lat}.
%
%
%
%
%
% - - - - - - - - - - -  Discretization  - - - - - - - - - - - -
%
%
%
%
%
\section{Discretizing the action}
In the previous steps we have constructed a \names{Lagrangian} fitting to our problem and modified terms to be able to apply a RHMC algorithm. To proceed with this task we need a discretized version of the bosonic action and the fermionic operator $\mathcal{O}_{\rm F}$.
%
%
%
% - - - - -  discretized bosnic action - - - - - - - -
%
%
%
\subsection{Bosonic action}
For the bosonic term this is quite easy. First we need to do a dimensional analysis of the fields in the action. In the simulation we can only deal with dimensionless fields. Since the action also needs to be dimensionless we can see that the fields $x,x^{*}$ and $z^{M}$ are dimensionless as preferred but the auxiliary fields are of dimension $[\phi]=1/[a]$. We therefore do a redefinition of the discretized fields
%
%
\begin{align}
a\phi(n) &\to \phi(n), & a\phi_{I}(n) &\to \phi_{I}(n)
\end{align}
%
%
in order to have dimensionless quantities as well. We can now write the discretized version of the bosonic action in (\ref{eq: bos_act}) as
%
%
\begin{align}
\hat{S}_{\rm B} &= g\, \sum\limits_{n\in\mathit{\Lambda}} \Bigg[ \left\vert x(n+\hat{0}) + \left(\tfrac{M}{2}-1\right) x(n) \right\vert^{2}
+ \frac{1}{[z(n)]^{4}} \left\vert x(n+\hat{1}) - \left(\tfrac{M}{2}+1\right) x(n) \right\vert^{2} \notag\\
%
%
&\qquad\qquad + \sum\limits_{M=1}^{6} \bigg\lbrace \left(z^{M}(n+\hat{0}) + \left(\tfrac{M}{2} -1\right)z^{M}(n) \right)^{2} \\
%
%
&\qquad\qquad\qquad\qquad + \frac{1}{[z(n)]^{4}} \left( z^{M}(n+\hat{1}) -\left(\tfrac{M}{2}+1\right) z^{M}(n) \right)^{2} \bigg\rbrace \notag \\
&\qquad\qquad +\phi^{2}(n) + \sum\limits_{I=1}^{16}\left[\phi_{I}(n)\right]^{2} \Bigg], \notag
\end{align}
%
%
where we applied a simple forward derivative to the $x,x^{*}$ and $z^{M}$ fields and introduced the dimensionless lattice mass parameter $M=ma$.
%
%
%
% - - - - - - - - - -  fermionic operator   - - - - -
%
%
%
%
%
\subsection{Wilson term}
Before discretizing the fermionic operator we have to think about the doubling problem arising through naively discretized first derivatives, discussed in \autoref{sec: ferm_doubling}. The free fermionic operator (evaluated in the bosonic vacuum) represented in a momentum basis reads
%
%
\begingroup
\everymath{\footnotesize}
\begin{flalign}
\!\!\!
K_{\rm F} &=
\begin{pmatrix}
0 & -p_{0}\mathds{1}_{4} & \left(p_{1}-i\tfrac{m}{2}\right)\rho^{M}u^{M} & 0 \\
-p_{0}\mathds{1}_{4} & 0 & 0 & \left(p_{1}-i\tfrac{m}{2}\right)\rho_{M}^{\dagger}u^{M} \\
-\left(p_{1}+i\tfrac{m}{2}\right)\rho^{M}u^{M} & 0 & 0 & -p_{0}\mathds{1}_{4} \\
0 & -\left(p_{1}+i\tfrac{m}{2}\right)\rho_{M}^{\dagger}u^{M} & -p_{0}\mathds{1}_{4} & 0
\end{pmatrix} ,
\raisetag{-8pt}
\end{flalign}
\endgroup
%
%
and has determinant
%
%
\begin{equation}
\det K_{\rm F} = \left( p_{0}^{2} + p_{1}^{2} + \frac{m^{2}}{4} \right)^{8}.
\label{eq: det_K_F}
\end{equation}
%
%
The fermionic propagators are proportional to the inverse of dynamic operators. It is therefore reasonable that the naive discretization (like in (\ref{eq: naive_first_dev}))
%
%
\begin{equation}
p_{i} \to \mathring{p}_{i} \equiv \frac{1}{a}\sin (p_{i}a)
\end{equation}
%
%
will give rise to doublers. For this reason we would like to introduce a \names{Wilson}-like term that cancels the additional poles in the fermionic propagator. It should obey the following conditions:
%
%
\begin{itemize}
\item preserve the maximal amount of symmetries and relevant matrix properties,
%
\item give the correct continuum limit for $a \to 0$,
%
\item should not give rise to a complex phase.
\end{itemize}
%
%
Due to the properties of $\mathcal{O}_{\rm F}$ presented in \autoref{sec: matrix_prop} we can see that there is only a small margin of variations that we can apply to the fermion matrix as a \names{Wilson} term and that respects the $U(1)$ and $SO(6)$ symmetries. In fact it was not possible to construct such an operator that also preserves relevant matrix properties like skew- and $\Gamma_{5}$-symmetry and also leads to the correct perturbative 1-loop coefficient in (\ref{eq: scaling_fct}). \\
We therefore chose to explicitly break the $U(1)$ symmetry and introduce a \names{Wilson}-like operator on the diagonal blocks of $\mathcal{O}_{\rm F}$. In terms of the free fermion operator in momentum space this takes the form
%
%
\begingroup
\everymath{\footnotesize}
\begin{flalign}
\!\!\!
\hat{K}_{\rm F} &=
\begin{pmatrix}
W_{+} & -\mathring{p}_{0}\mathds{1}_{4} & \left(\mathring{p}_{1}-i\tfrac{m}{2}\right)\rho^{M}u^{M} & 0 \\
-\mathring{p}_{0}\mathds{1}_{4} & -W_{+}^{\dagger} & 0 & \left(\mathring{p}_{1}-i\tfrac{m}{2}\right)\rho_{M}^{\dagger}u^{M} \\
-\left(\mathring{p}_{1}+i\tfrac{m}{2}\right)\rho^{M}u^{M} & 0 & W_{-} & -\mathring{p}_{0}\mathds{1}_{4} \\
0 & -\left(\mathring{p}_{1}+i\tfrac{m}{2}\right)\rho_{M}^{\dagger}u^{M} & -\mathring{p}_{0}\mathds{1}_{4} & -W_{-}^{\dagger}
\end{pmatrix} ,
\raisetag{-8pt}
\end{flalign}
\endgroup
%
%
where
%
%
\begin{equation}
W_{\pm} = \frac{ra}{2} \left( \hat{p}_{0}^{2} \pm i \hat{p}_{1}^{2} \right) \rho^{M}u^{M}, \qquad \vert r \vert =1,
\end{equation}
%
%
and similar to (\ref{eq: wilson_op})
%
%
\begin{equation}
\hat{p}_{i} \equiv \frac{2}{a} \sin \frac{p_{i}a}{2}\,.
\end{equation}
%
%
This leads to the analogue expression of (\ref{eq: det_K_F})
%
%
\begin{equation}
\det \hat{K}_{\rm F} = \left( \mathring{p}_{0}^{2} + \mathring{p}_{1}^{2} +\frac{r^{2}a^{2}}{4}\left( \hat{p}_{0}^{4}+\hat{p}_{1}^{4} \right) + \frac{m^{2}}{4} \right)^{8}\,.
\end{equation}
%
%
If we substitute this into the denominator of (\ref{eq: 1_loop}) and apply the replacement $p_{i}^{2} \to \hat{p}_{i}^{2}$ in the numerator, the discretized equivalent of the 1-loop free energy can be defined by
%
%
\begin{equation}
\Gamma^{(1)}_{\rm LAT} = -\ln Z^{(1)}_{\rm LAT} = \mathcal{I}(a).
\end{equation}
%
%
With $r=1$ and rescaled momentum integration over the first \names{Brillouine} zone this results in
%
%
\begin{align}
\mathcal{I}(a) = \frac{V_{2}}{2a^{2}} &\int\limits_{-\pi}^{\pi} \frac{\dd^{2}p}{(2\pi)^{2}} \Bigg\lbrace 5\ln \bigg[ 4 \Big(\sin^{2}\frac{p_{0}}{2} + \sin^{2}\frac{p_{1}}{2}\Big)\bigg] \notag \\
%
%
&+2\ln \bigg[ 4 \Big(\sin^{2}\frac{p_{0}}{2} + \sin^{2}\frac{p_{1}}{2} + \frac{M^{2}}{8} \Big)\bigg] \label{eq: I_a} \\
%
%
&+\ln \bigg[ 4 \Big(\sin^{2}\frac{p_{0}}{2} + \sin^{2}\frac{p_{1}}{2} + \frac{M^{2}}{4} \Big)\bigg]  \notag\\
%
%
&-8 \ln\bigg[ 4 \sin^{4}\frac{p_{0}}{2} +\sin^{2} p_{0} + 4\sin^{4}\frac{p_{1}}{2} + \sin^{2} p_{1} + \frac{M^{2}}{4}\bigg] \Bigg\rbrace. \notag
\end{align}\\[0.2cm]
%
%
For a consistent discretization this should lead to the same result as (\ref{eq: 1_loop}) in the $a \to 0$ limit. And indeed a numerical integration of (\ref{eq: I_a}) yields
%
%
\begin{equation}
\Gamma^{(1)} = -\ln Z^{(1)} = \lim\limits_{a\to 0} \mathcal{I}(a) = -\frac{3\ln 2}{8 \pi} \vert \mathit{\Lambda}\vert M^{2},
\end{equation}
%
%
where we used that $V_{2}=a^{2}\vert \mathit{\Lambda}\vert = a^{2}LT$ and we are left with the expected result. Given the structure of the \names{Wilson} term in the vacuum it is quite natural to generalize it to the interacting case. The discretized momentum space operator therefore reads
%
%
\begingroup
\everymath{\footnotesize}
\begin{flalign}
\!\!\!\!\!\!\!\!\!
\widetilde{\mathcal{O}}_{\rm F} &=
\begin{pmatrix}
W_{+} & -\mathring{p}_{0}\mathds{1}_{4} & \left(\mathring{p}_{1}-i\tfrac{m}{2}\right)\rho^{M}u^{M} & 0 \\
-\mathring{p}_{0}\mathds{1}_{4} & -W_{+}^{\dagger} & 0 & \left(\mathring{p}_{1}-i\tfrac{m}{2}\right)\rho_{M}^{\dagger}u^{M} \\
-\left(\mathring{p}_{1}+i\tfrac{m}{2}\right)\rho^{M}u^{M} & 0 & \!\!\!\!\!\!\!\!\!\!\!\!    2\tfrac{z^{M}}{z^{4}}\rho^{M}(\del_{s}x-\tfrac{m}{2}x) + W_{-} & -\mathring{p}_{0}\mathds{1}_{4} \\
0 &  \!\!\!\!\!\!\!\!\! -\left(\mathring{p}_{1}+i\tfrac{m}{2}\right)\rho_{M}^{\dagger}u^{M} & -\mathring{p}_{0}\mathds{1}_{4} &\!\!\!\!\!\!\!\!\!\!\!\!    -2\tfrac{z^{M}}{z^{4}}\rho_{M}^{\dagger}(\del_{s}x^{*}-\tfrac{m}{2}x^{*})-W_{-}^{\dagger}
\end{pmatrix} ,
\raisetag{-8pt}
\end{flalign}
\endgroup
%
%
with
%
%
\begin{equation}
W_{\pm} = \frac{ra}{2z^{2}}\left(\hat{p}_{0}^{2} \pm i\hat{p}_{1}^{2}\right)\rho^{M}z^{M},
\end{equation}
%
%
where an additional $1/z$ factor is present for the purpose of improved stability during simulations. From (\ref{eq: G5_herm_mat}) one can see that the added \names{Wilson} term respects the $\Gamma_{5}$-hermiticity and skew-symmetry which ensures the determinant to be real and positive.
%
%
%
%
%
% - - - - - - - - - - - -  discretizing the fermionic operator   - - - - - - - - - - - - -
%
%
%
%
%
\subsection{Fermionic operator}
\label{sec: ferm_op}
By knowing the structure of the \names{Wilson} term we can finally discretize the fermionic operator and write it in terms of a single matrix by using the lexicographic index notation introduced in \autoref{sec: disc_lat}. The discretized fermion matrix $\hat{\mathcal{O}}_{\rm F}$ is of size $16V\times 16V$ and we are going to subdivide it into 4 by 4 blocks of size $4V\times 4V$ as
%
%
\begin{equation}
\left(\hat{\mathcal{O}}_{\rm F}\right)_{16V\times 16V} =
\left(
\begin{array}{c|c|c|c}
O_{4V\times 4V} & \cdots & \cdots & \cdots \\ \hline
\vdots & \ddots & \qquad & \qquad \\ \hline
\vdots & \qquad & \ddots & \qquad \\ \hline
\vdots & \qquad & \qquad & \ddots
\end{array}
\right).
\end{equation}
%
%
To build the block matrices we assume that for every lattice index $l,p=0,\ldots,(V-1)$ we have a $4\times 4$ matrix with indices $i,j=1,\ldots,4$ and emphasize the structure
%
%
\begin{equation}
O_{4\times 4}(V\times V) \longleftrightarrow O_{ij}(l,p).
\end{equation}
%
%
One can map this to a $4V\times 4V$ matrix with global indices like
%
%
\begin{equation}
O_{4V\times 4V} \longleftrightarrow O_{AB}, \qquad A=4l+i,\quad B=4p+j.
\end{equation}
%
%
The discretized fermion matrix is now given by
%
%
\begingroup
\everymath{\footnotesize}
\begin{flalign}
\!
\hat{\mathcal{O}}_{\rm F} =
\begin{pmatrix}
\hat{W}_{+} & i\bar{\Delta}^{\rm A}_{0} & -i\left(\vec{\Delta}^{Z}_{1} + \tfrac{M}{2}\bar{Z}\right) & 0 \\
i\bar{\Delta}^{\rm A}_{0} & -\hat{W}_{+}^{\dagger} & 0 & -i\left(\vec{\Delta}^{Z^{\dagger}}_{1} + \tfrac{M}{2}\bar{Z}^{\dagger}\right)  \\
i\left(\cev{\Delta}^{Z}_{1} - \tfrac{M}{2}\bar{Z}\right)  & 0 & \!\!\!\!\!\!\!\!\! 2\left(\bar{\Delta}^{x}_{1}-\tfrac{M}{2}\bar{Z}^{x}\right)+\hat{W}_{-} & i\bar{\Delta}^{\rm A}_{0} -\hat{A}^{\rm T} \\
0 & i\left(\cev{\Delta}^{Z^{\dagger}}_{1} - \tfrac{M}{2}\bar{Z}^{\dagger}\right) & i\bar{\Delta}^{\rm A}_{0} +\hat{A} & \!\!\!\!\! 2\left(\bar{\Delta}^{x^{*}}_{1}-\tfrac{M}{2}\bar{Z}^{x^{*}}\right)-\hat{W}_{-}^{\dagger}
\end{pmatrix} ,
\raisetag{-8pt}
\label{eq: disc_OF}
\end{flalign}
\endgroup
%
%
where we refer the reader to Appendix \ref{app: disc_ferm} for a detailed derivation of the single terms.
%
%
%
%
%
% - - - - - - - - - - - -    Applying the RHMC   - - - - - - - - - - 
%
%
%
%
%
\section{Applying the RHMC algorithm}
Since the current status is a realization of discretized operators and fields that are also dimensionless, we can return to a discretized version of the equations (\ref{eq: Z_string2}-\ref{eq: S_xi}) and implement a RHMC algorithm based on HMC algorithm introduced in section \ref{sec: hmc_alg}. There we described the algorithm by means of an inversed hermitian operator in the pseudofermionic action
%
%
\begin{equation}
\hat{S}_{\xi} = g \xi^{\dagger}\left(\hat{M}^{\dagger}\hat{M}\right)^{-1}\xi,
\end{equation}
%
%
where we applied a matrix-vector notation in the sense of (\ref{eq: lexi_mat_vec}). In our present case we are instead faced with
%
%
\begin{equation}
\hat{S}_{\xi} = g \xi^{\dagger}\left(\hat{\mathcal{O}}_{\rm F}^{\dagger}\hat{\mathcal{O}}_{\rm F}\right)^{-\frac{1}{4}} \xi.
\end{equation}
%
%
To treat this kind of matrix valued function one uses a rational approximation
%
%
\begin{equation}
\left(\hat{M}^{\dagger}\hat{M}\right)^{\rho	} = \alpha_{0} + \sum\limits_{i=1}^{P} \frac{\alpha_{i}}{\hat{M}^{\dagger}\hat{M}+\beta_{i}\mathds{1}},
\label{eq: rat_approx}
\end{equation}
%
%
where $\alpha_{i}$ and $\beta_{i}$ are constant coefficients. For predicted limits of the eigenvalues $\lambda_{\rm min}$ and $\lambda_{\rm max}$ of a matrix and of cause a given exponent $\rho$ one can use a \names{Remez} algorithm to obtain the coefficients $\alpha_{i}$ and $\beta_{i}$ up to some order $P$ of accuracy of the approximation, valid for any matrix satisfying the spectral bounds.\\
For the implementation of the RHMC we need two sets of coefficients for two different approximations with $\rho=1/8$ and $\rho=-1/4$, respectively. The requirement of $\rho=-1/4$ should be clear, but we also need an approximation with $\rho=1/8$ to generate pseudofermionic fields\footnote{Compare to the list of items in section \ref{sec: hmc_alg}.} 
%
%
\begin{equation}
\xi = \left(\hat{\mathcal{O}}_{\rm F}^{\dagger}\hat{\mathcal{O}}_{\rm F}\right)^{\frac{1}{8}} \vartheta
\end{equation}
%
%
distributed according to
%
%
\begin{equation}
\exp\left(-\vartheta^{\dagger}\vartheta\right) = \exp\left( -\xi^{\dagger}\left(\hat{\mathcal{O}}_{\rm F}^{\dagger}\hat{\mathcal{O}}_{\rm F}\right)^{-\frac{1}{4}} \xi \right).
\end{equation}
%
%
We use an order $P=15$ approximation and the utilized values $\alpha_{i}$ and $\beta_{i}$ can be found in \autoref{tab: rat_app_coef}. In order to state the full RHMC procedure we want to take  a closer look at the pseudofermionic contribution to the bosonic force
%
%
\begin{equation}
F_{m}^{\xi}[\mathit{\Phi}] = \frac{\del }{\del \mathit{\Phi}_{m}}\left[ \xi^{\dagger}\left(\hat{\mathcal{O}}_{\rm F}^{\dagger}\hat{\mathcal{O}}_{\rm F}\right)^{-\frac{1}{4}} \xi \right], \quad \mathit{\Phi}_{m} = \left(x,x^{*},z^{N},\phi,\phi^{I}\right),
\end{equation}
%
%
with $\quad m=1,\ldots,25$. Inserting the rational approximation yields\footnote{With help of the matrix-scalar derivative 
$\dfrac{\del U^{-1}(x)}{\del x} = -U^{-1}\dfrac{\del U}{\del x} U^{-1}.$ \vspace{0.2cm}}
%
%
\begin{equation}
\begin{alignedat}{2}
F_{m}^{\xi}[\mathit{\Phi}] &= - \sum\limits_{i=1}^{P} \alpha_{i} \left(\frac{1}{\hat{\mathcal{O}}_{\rm F}^{\dagger}\hat{\mathcal{O}}_{\rm F}+\beta_{i}\mathds{1}} \xi \right)^{\dagger} \frac{\del }{\del \mathit{\Phi}_{m}}\left(\hat{\mathcal{O}}_{\rm F}^{\dagger}\hat{\mathcal{O}}_{\rm F}\right) \left(\frac{1}{\hat{\mathcal{O}}_{\rm F}^{\dagger}\hat{\mathcal{O}}_{\rm F}+\beta_{i}\mathds{1}} \xi \right) \\
%
%
&= -\sum\limits_{i=1}^{P} \alpha_{i}\left[ \left(\hat{\mathcal{O}}_{\rm F} s_{i}\right)^{\dagger} \frac{\del \hat{\mathcal{O}}_{\rm F}}{\del \mathit{\Phi}_{m}} s_{i} + s_{i}^{\dagger} \frac{\del \hat{\mathcal{O}}_{\rm F}^{\dagger}}{\del \mathit{\Phi}_{m}} \hat{\mathcal{O}}_{\rm F} s_{i} \right],
\end{alignedat}
\end{equation}
%
%
where $s_{i}= \left(\hat{\mathcal{O}}_{\rm F}^{\dagger}\hat{\mathcal{O}}_{\rm F}+\beta_{i}\mathds{1}\right)^{-1}\xi$ are interpreted as the solutions to the matrix equation
%
%
\begin{equation}
\left(\hat{\mathcal{O}}_{\rm F}^{\dagger}\hat{\mathcal{O}}_{\rm F}+\beta_{i}\mathds{1}\right) s^{i} = \xi, \qquad i=1,\ldots,P \;.
\end{equation}
%
%
To obtain all $P$ solutions at once, we employ a \textit{multi-shift conjugate gradient solver} adapted from the \texttt{openQCD} package\footnote{\url{http://luscher.web.cern.ch/luscher/openQCD/}} by Martin Lüscher. For the single terms of the matrix derivative $\del \hat{\mathcal{O}}_{\rm F} / \del \mathit{\Phi}_{m}$ we refer the reader to the Appendix \textbf{{\LARGE INSERT APP REF}}. The single steps for the RHMC are therefore:
%
%
\begin{itemize}
\item \textbf{Pseudofermions}\\
Generate the $16V$-dimensional pseudofermionic field $\xi = \left(\hat{\mathcal{O}}_{\rm F}^{\dagger}\hat{\mathcal{O}}_{\rm F}\right)^{\frac{1}{8}} \vartheta$, where $\vartheta$ is distributed according $\exp\left(-\vartheta^{\dagger}\vartheta\right)$.
%
%
\item \textbf{Conjugate fields}\\
For an initial boson configuration $\mathit{\Phi}^{(0)}$ generate $\mathit{\Pi}^{(0)}$ according to the \names{Gaussian} distribution $\exp \left(-\frac{1}{2}\mathit{\Pi}^{2}\right)$.
%
%
\item \textbf{Initial step}\\
$\mathit{\Pi}_{m}^{(\frac{1}{2})} = \mathit{\Pi}_{m}^{(0)} - \frac{\epsilon}{2} F_{m}[\mathit{\Phi}]\Big\vert_{\mathit{\Phi}^{(0)}}  \;.$
%
%
\item \textbf{Intermediate step}\\
Full steps for $j=1,\ldots,n-1$ \\[0.2cm]
$\mathit{\Phi}_{m}^{(j)} = \mathit{\Phi}_{m}^{(j-1)} + \mathit{\Pi}_{m}^{(j-\frac{1}{2})}, \quad  \mathit{\Pi}_{m}^{(j+\frac{1}{2})} = \mathit{\Pi}_{m}^{(j-\frac{1}{2})} - \epsilon F_{m}[\mathit{\Phi}]\Big\vert_{\mathit{\Phi}^{(j)}}\;.$
%
%
\item \textbf{Final step}\\
$\mathit{\Phi}_{m}' = \mathit{\Phi}_{m}^{(n)} = \mathit{\Phi}_{m}^{(n-1)} + \mathit{\Pi}_{m}^{(n-\frac{1}{2})}, \quad \mathit{\Pi}_{m}' = \mathit{\Pi}_{m}^{(n)} = \mathit{\Pi}_{m}^{(n-\frac{1}{2})} - \frac{\epsilon}{2} F_{m}[\mathit{\Phi}]\Big\vert_{\mathit{\Phi}^{(n)}}\;.$
%
%
\item \textbf{Monte Carlo step}\\
Accept new configuration $\mathit{\Phi}'$ if a random number $r \in [0,1)$ is smaller than
\[\exp\left(\frac{1}{2}\left(\mathit{\Pi}^{2}-\mathit{\Pi}'^{2}\right) + S_{\rm B}[\mathit{\Phi}]- S_{\rm B}[\mathit{\Phi}'] +\vartheta^{\dagger}\vartheta - \xi^{\dagger} \left(\hat{\mathcal{O}}_{\rm F}^{\dagger}\hat{\mathcal{O}}_{\rm F}\right)^{-\frac{1}{4}} \xi \right) \;.  \]
\end{itemize}
%
%
Hereby we have used 
%
%
\begin{equation}
F_{m}[\mathit{\Phi}] = \frac{\del S_{\rm B}[\mathit{\Phi}]}{\del \mathit{\Phi}_{m}} + F_{m}^{\xi}[\mathit{\Phi}].
\end{equation}