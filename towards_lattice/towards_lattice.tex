\section{Towards the lattice simulation}
\label{sec: towards_lat}
With the current status of the action (\ref{eq: cusp_action}) we could almost start to discretize the operators and fields, at least for the bosonic part this would not be a problem. For the fermions however this is not so straight forward. In order to include the fermionic contribution into the weight factor of the path integral like explained in section \ref{sec: hmc_alg} one needs to integrate out the \names{Grassmann} variables to result into a \names{Pfaffian} or determinant of a fermionic operator. As presented in Appendix \ref{sec: grassmann_analysis} this is only possible if the fermions appearing are of quadratic order. But in the fluctuation action (\ref{eq: cusp_action}) also quartic contributions of fermions appear, which have to be linearized with help of a \names{Hubbard-Statonovich} transformation.\footnote{See Appendix \ref{sec: grassmann_analysis} for details}
%
%
%
% - - - - - - - - -  linearization of the action  - - - - - - - - - - - -
%
%
%
\subsection{Linearization of fermionic contributions}
\subsubsection{Naive approach and sign problem}
The only quartic interactions are coming from the $\eta$ fields and we can write this part of the action as
%
%
\begin{equation}
S^{\rm F}_{4} [\eta_{i},\eta^{i}] = g \int \dd t \dd s \, \left[-\frac{1}{z^{2}}\left(\eta^{2}\right)^{2} + \left(\frac{i}{z^{2}}z_{N}
\eta_{i}{\left(\rho^{MN}\right)^{i}}_{j}\eta^{j}\right)^{2} \right].
\label{eqq: S_4}
\end{equation}
%
%
In the path integral representation the euclidean action contributes within an exponential $e^{-S_{\rm E}}$. By performing a naive \names{Hubbard-Statonovich} transformation to this exponential we can reduce the four-fermion contributions to quadratic \names{Yukawa} terms whereas we have to introduce 7 bosonic real auxiliary fields $\phi$ and $\phi^{N}$
%
%
\begin{align}
\exp \bigg\lbrace -g \int \dd t \dd s \, \bigg[-\tfrac{1}{z^{2}} \big(\eta^{2}\big)^{2} + \Big(\tfrac{i}{z^{2}}z_{N}
& \eta_{i}{\big(\rho^{MN}\big)^{i}}_{j}\eta^{j}\Big)^{2} \bigg] \bigg\rbrace \\
%
%
\sim \int \mathcal{D}\phi \mathcal{D}\mathcal{\phi^{M}}\; \exp\bigg\lbrace -g \int \dd t \dd s \, \bigg[ &\tfrac{1}{2}\phi^{2} + \tfrac{\sqrt{2}}{z}\phi\eta^{2} +\tfrac{1}{2} (\phi^{M})^{2} \notag \\
 &- i\tfrac{\sqrt{2}}{z^{4}} \phi_{M} \left( i\eta_{i}{(\rho^{MN})^{i}}_{j}\eta^{j}\right) z_{N} \bigg] \bigg\rbrace. \notag
\end{align}
%
%
Here we can notice that the second term appears to be complex, since the $SO(6)$ matrix in parenthesis is hermitian (with respect to the indices $M,N$)
%
%
\begin{equation}
\left( i \,\eta_{i} {\left(\rho^{MN}\right)^{i}}_{j}\eta^{j}\right)^{\dagger} = i\, \eta_{j} {\left(\rho^{MN}\right)^{j}}_{i}\eta^{i},
\end{equation}
%
%
where we have used (\ref{eq: SO6_id}). As discussed in section \ref{sec: sign_prob} this complex phase in the weight function is potentially leading to a non treatable sign problem. We therefore chose to make a field redefinition that circumvents the appearance of a complex phase during the HS transformation. 
%
%
%
% - - - - alternative field def - - - - - 
%
%
%
\subsubsection{Alternative field redefinition}
By using the identities for the $SO(6)$ matrices stated in Appendix \ref{sec: SO6} we can rewrite the second term in the \names{Laggrangian} of (\ref{eqq: S_4}) as
%
%
\begin{equation}
\left(i\, \eta_i {(\rho^{MN})^i}_j n^N \eta^j\right)^2=-3 (\eta^2)^2+2\eta_i (\rho^N)^{ik} n_N \eta_k \eta^j (\rho^L)_{jl} n_L \eta^l,
\end{equation}
%
%
where we defined $n^{N}=\tfrac{z^{N}}{z}$. This leads to the \names{Lagrangian}
%
%
\begin{equation}
\mathcal{L}_4=\frac{1}{z^2}\left(- 4\, (\eta^2)^2+2\left|\eta_i (\rho^N)^{ik} n_N \eta_k\right|^2\right).
\end{equation}
%
%
In order to circumvent the sign problem the second term needs to be negative. To achieve this we define new fields\footnote{Where we actually set ${\Sigma_i}^j=\eta_i \eta^j$, then defined ${\Sigma^{i}}_{j}\equiv({\Sigma_i}^{j})^{*}={\Sigma_{j}}^{i}$ to emphasize the notation $\Sigma_i^j$ and equivalent for $\tilde{\Sigma}$.}
%
%
\begin{equation}
\Sigma_i^j=\eta_i \eta^j \qquad \tilde{\Sigma}_j^i=(\rho^N)^{ik}n_N (\rho^L)_{jl}n_L \eta_k \eta^l.
\end{equation}
%
%
with this new definitions it is simple to check that
%
%
\begin{align}
 \Sigma^j_{i}\Sigma^i_j&=-(\eta^2)^2 & \tilde\Sigma^j_{i} \tilde\Sigma^i_j&=-(\eta^2)^2 & \Sigma^i_j \tilde\Sigma^j_i&=-\left|\eta_i (\rho^N)^{ik} n_N \eta_k\right|^2.
\end{align}
%
%
With this we now define
%
%
\begin{equation}
{\Sigma_{\pm}}_i^j=\Sigma_i^j\pm \tilde \Sigma_i^j
\end{equation}
%
%
and find
%
%
\begin{equation}
 {\Sigma_{\pm}}_i^j{\Sigma_{\pm}}_j^i=-2(\eta^2)^2 \mp 2\left|\eta_i (\rho^N)^{ik} n_N \eta_k\right|^2.
\end{equation}
%
%
We can now substitute the new fields into the \names{Lagrangian} and obtain
%
%
\begin{equation}
\mathcal{L}_4=\frac{1}{z^2}\left(- 4\, (\eta^2)^2 \mp 2(\eta^2)^2 \mp {\Sigma_{\pm}}_i^j{\Sigma_{\pm}}_j^i \right)\,,
\end{equation}
%
%
where we only need to select the right sign in the field definition to overcome the sign problem, which is leading to
%
%
\begin{equation}
\mathcal{L}_4= \frac{1}{z^2}\left(- 6\, (\eta^2)^2 - {\Sigma_{+}}_i^j{\Sigma_{+}}_j^i \right)\,.
\end{equation}
%
%
If we now perform a HS transformation there will be no complex phase. The HS transformation yields
%
%
\begin{equation}
 -\frac{6}{z^2}(\eta^2)^2\to \frac{12}{z} \eta^2 \phi +6\phi^2,
 \label{eq: HS_phi}
\end{equation}
%
%
where a single bosonic field was introduced like in the naive case. And further
%
%
\begin{equation}
-\frac{1}{z^{2}}{\Sigma_{+}}^{i}_{j}{\Sigma_{+}}^{j}_{i} \rightarrow \frac{2}{z}{\Sigma_{+}}^{i}_{j}\phi^{j}_{i} + \phi^{i}_{j}\phi^{j}_{i}
\qquad \text{with} \qquad \left(\phi^{i}_{j}\right)^{\ast} = \phi^{j}_{i}\;.
\end{equation}
%
%
Here the collection of fields $\phi^{i}_{j}$ can be thought of as a complex hermitian matrix with 16 real free parameters. We find it convenient to rescale the field $\phi \to \phi / \sqrt{6}$, to get rid of the pre factor of 6 in (\ref{eq: HS_phi}). After reinserting the old fields for $\Sigma_{+}$ we can conclude that
%
%
\begin{equation}
 \mathcal{L}_4\to \frac{12}{z} \eta^2 \phi +\phi^2+\frac{2}{z}\eta_j \phi^j_i \eta^i +\frac{2}{z} (\rho^N)^{ik}n_N \eta_k\phi^j_i  (\rho^L)_{jl}n_L  \eta^l+\phi^i_j \phi^j_i\,. 
\end{equation}
%
%
So now we can write the full \names{Lagrangian} as 
%
%
\begin{align}
\mathcal{L}_{\rm cusp} &=  {\left\vert \partial_t {x} + {\frac{m}{2}}{x} \right\vert}^2 + \frac{1}{{ z}^4}{\left\vert \partial_s {x} -\frac{m}{2}{x} \right\vert}^2 + \left(\partial_t {z}^M + \frac{m}{2}{z}^M \right)^2 \\ &\quad+ \frac{1}{{ z}^4} \left(\partial_s {z}^M -\frac{m}{2}{z}^M\right)^2 
+ \phi^2 + {\rm Tr} \left( \tilde{\phi}\, \tilde{\phi}^{\dagger} \right) + \mathit{\Psi}^T \mathcal{O}_{\rm F} \mathit{\Psi} \,. \notag
\end{align}
%
%
Hereby we defined the fermionic vector $\mathit{\Psi}\equiv (\theta^{i},\theta_{i},\eta^{i},\eta_{i})$ as well as the auxiliary matrix ${\tilde{\phi}= (\tilde{\phi}_{ij}) \equiv \phi^{i}_{j}}$. We used partial integration and the properties of the \names{Grassmann} numbers and $SO(6)$ matrices to write the fermionic contribution in a matrix-vector notation. The $16\times 16$ fermionic operator is hereby represented as $4\times 4$ block matrix
%
%
\begingroup
\everymath{\footnotesize}
%\normalsize
\begin{flalign} \label{OF}
\!\!\!\!\!\!\!\!
\mathcal{O}_{\rm F} & =\begin{pmatrix}
0 & i \mathds{1}_{4}\partial_{t} & -i\rho^{M}\left(\partial_{s}+\frac{m}{2}\right)\frac{{z}^{M}}{{z}^{3}} & 0\\
i \mathds{1}_{4}\partial_{t} & 0 & 0 & -i\rho_{M}^{\dagger}\left(\partial_{s}+\frac{m}{2}\right)\frac{{z}^{M}}{{z}^{3}}\\
i\frac{{z}^{M}}{{z}^{3}}\rho^{M}\left(\partial_{s}-\frac{m}{2}\right) & 0 & 2\frac{{z}^{M}}{{z}^{4}}\rho^{M}\left(\partial_{s}{x}-m\frac{{x}}{2}\right) & i \mathds{1}_{4}\partial_{t}-A^{T}\\
0 & i\frac{{z}^{M}}{{z}^{3}}\rho_{M}^{\dagger}\left(\partial_{s}-\frac{m}{2}\right) & i \mathds{1}_{4}\partial_{t}+A & -2\frac{{z}^{M}}{{z}^{4}}\rho_{M}^{\dagger}\left(\partial_{s}{x}^\ast-m\frac{{x}}{2}^\ast\right)
\end{pmatrix}~,
\raisetag{-8pt}
\end{flalign}
\endgroup
%
%
where
%
%
\begin{equation}
A=-\frac{\sqrt{6}}{z}\phi \mathds{1}_{4} + \frac{1}{z}\tilde{\phi}+\frac{1}{z^{3}}\rho^\ast_{N}\tilde{\phi}^{T}\rho^{L}z^{N}z^{L}+\mathrm{i}\frac{z^{N}}{z^2}\rho^{MN}\partial_{t}z^{M}.
\end{equation}
%
%
The auxiliary matrix $\tilde{\phi}$ is constructed from 16 real auxiliary fields $\phi_{I}\; (I=1,\ldots,16)$ in the following way
%
%
\begin{equation}
\tilde{\phi} = \frac{1}{\sqrt{2}}
\begin{pmatrix}
\sqrt{2}\phi_{13} & \phi_{1}+i\phi_{2} & \phi_{3}+i\phi_{4} & \phi_{5}+i\phi_{6} \\ 
\phi_{1}-i\phi_{2} & \sqrt{2}\phi_{14} & \phi_{7}+i\phi_{8} & \phi_{9}+i\phi_{10} \\ 
\phi_{3}-i\phi_{4} & \phi_{7}-i\phi_{8} & \sqrt{2}\phi_{15} & \phi_{11}+i\phi_{12} \\ 
\phi_{5}-i\phi_{6} & \phi_{9}-i\phi_{10} & \phi_{11}-i\phi_{12} & \sqrt{2}\phi_{16}
\end{pmatrix} ,
\end{equation}
%
%
so that we have the simple expression for
%
%
\begin{equation}
{\rm Tr}\left(\tilde{\phi}\, \tilde{\phi}^{\dagger}\right) = \sum\limits_{I=1}^{16} (\phi_{I})^{2} \equiv (\phi_{I})^{2}\,.
\end{equation}
%
%
%
%
%
% - - - - - - - - pseudofermions - - - - - - - - - - -
%
%
%
%
%
\subsubsection{Pseudofermionic weight function}
Now since we have linearised fermionic contributions to quadratic order, we are able to integrate over the \names{Grassmann} fields in the partition function
%
%
\begin{equation}
Z = \int \mathcal{D}x\,\mathcal{D}x^{*}\,\mathcal{D}z^{N}\,\mathcal{D}\phi\,\mathcal{D}\phi_{I}\,\mathcal{D}\mathit{\Psi}\; e^{-S}\,,
\end{equation}
%
%
where we will split $S = S_{\rm B} + S_{\rm F}$ into its bosonic ($S_{\rm B}$) and fermionic ($S_{\rm F}$) contributions with
%
%
\begin{align}
S_{\rm B} &= g\,\int \dd t\dd s\; {\left\vert \partial_t {x} + {\frac{m}{2}}{x} \right\vert}^2 + \frac{1}{{ z}^4}{\left\vert \partial_s {x} -\frac{m}{2}{x} \right\vert}^2 + \left(\partial_t {z}^M + \frac{m}{2}{z}^M \right)^2 \notag \\ 
&\qquad\qquad+ \frac{1}{{ z}^4} \left(\partial_s {z}^M -\frac{m}{2}{z}^M\right)^2 + \phi^2 + (\phi_{I})^{2}\,, \label{eq: bos_act} \\[0.5cm]
S_{\rm F} &=  g\,\int \dd t\dd s\; \mathit{\Psi}^T \mathcal{O}_{\rm F} \mathit{\Psi}\,. \notag 
\end{align}
%
%
As motivated in section \ref{sec: grassmann_analysis} the \names{Grassmann} integral over $\mathit{\Psi}$ will result in a \names{Pfaffian} ${\rm Pf}\,\mathcal{O}_{\rm F}$. To include the \names{Pfaffian} into the weight function we have to rewrite it in terms of pseudofermions $\xi$ as emphasized in section \ref{sec: hmc_alg}. To legitimately apply this procedure we need the \names{Pfaffian} to be real and non-negative. By our alternative approach to linearization we made sure to exclude any potential phase ambiguities. All terms in the action $S_{\rm F}$ are real and therefore also the \names{Pfaffian} turns out to be a real quantity. So the only problem that we might have to face is if the \names{Pfaffian} is negative. We would need to be able to write the \names{Pfaffian} as a perfect square of another quantity to show that it is real for all configurations. We could not show this to be the case analytically (except for some special cases) but there are some hints that point out for this to be legitimate to assume. First of all it can be observed numerically that the \names{Pfaffian} is always positive at least for large values of $g$ and second also the eigenvalue spectrum of the fermion matrix suggests this, but we will postpone this argument to a later point when we have discretized the operator $\mathcal{O}_{\rm F}$. For now we will assume a positive \names{Pfaffian} for all situations and make this more legitimate by keeping track of it during simulations. For this reason, we proceed as in \cite{Roiban} and introduce pseudofermions $\xi$ via
%
%
\begin{equation}
\int \mathcal{D}\mathit{\Psi}\; e^{-S_{\rm F}} = {\rm Pf}\,\mathcal{O}_{\rm F} =\left( \det \mathcal{O}_{\rm F}\,\mathcal{O}_{\rm F}^{\dagger} \right)^{\frac{1}{4}} = \int \mathcal{D}\xi\,\mathcal{D}\xi^{\dagger}\; e^{-S_{\xi}}\,,
\end{equation}
%
%
where
%
%
\begin{equation}
S_{\xi} = g\,\int \dd t\dd s \; \xi^{\dagger} \left(\mathcal{O}_{\rm F}\,\mathcal{O}_{\rm F}^{\dagger} \right)^{-\frac{1}{4}} \xi \,,
\end{equation}
and $\xi$ is a collection of 16 complex bosonic scalar fields. But before we can go any further we need to discretize the action with help of the methods introduced in \autoref{sec: disc_lat}. 
%
%
%
%
%
% - - - - - - - - - - -  Discretization  - - - - - - - - - - - - 
%
%
%
%
%
\subsection{Discretizing the action}
In the previous steps we have constructed a \names{Lagrangian} fitting to our problem and modified terms to be able to apply a RHMC algorithm. To proceed with this task we need a discretized version of the bosonic action and the fermionic operator $\mathcal{O}_{\rm F}$. 
%
%
%
% - - - - -  discretized bosnic action - - - - - - - -
%
%
%
\subsubsection{Bosonic action}
For the bosonic term this is quite easy. First we need to do a dimensional analysis of the fields in the action. In the simulation we can only deal with dimensionless fields. Since the action also needs to be dimensionless we can see that the fields $x,x^{*}$ and $z^{M}$ are dimensionless as preferred but the auxiliary fields are of dimension $[\phi]=1/[a]$. We therefore do a redefinition of the discretized fields
%
%
\begin{align}
a\phi(n) &\to \phi(n), & a\phi_{I}(n) &\to \phi_{I}(n)
\end{align}
%
%
in order to have dimensionless quantities as well. We can now write the discretized version of the bosonic action in (\ref{eq: bos_act}) as
%
%
\begin{align}
\hat{S}_{\rm B} &= g\, \sum\limits_{n\in\mathit{\Lambda}} \Bigg[ \left\vert x(n+\hat{0}) + \left(\tfrac{M}{2}-1\right) x(n) \right\vert^{2} 
+ \frac{1}{[z(n)]^{4}} \left\vert x(n+\hat{1}) - \left(\tfrac{M}{2}+1\right) x(n) \right\vert^{2} \notag\\
%
%
&\qquad\qquad + \sum\limits_{M=1}^{6} \bigg\lbrace \left(z^{M}(n+\hat{0}) + \left(\tfrac{M}{2} -1\right)z^{M}(n) \right)^{2} \\
%
%
&\qquad\qquad\qquad\qquad + \frac{1}{[z(n)]^{4}} \left( z^{M}(n+\hat{1}) -\left(\tfrac{M}{2}+1\right) z^{M}(n) \right)^{2} \bigg\rbrace \notag \\
&\qquad\qquad +\phi^{2}(n) + \sum\limits_{I=1}^{16}\left[\phi_{I}(n)\right]^{2} \Bigg], \notag
\end{align}
%
%
where we applied a simple forward derivative to the $x,x^{*}$ and $z^{M}$ fields and introduced the dimensionless lattice mass parameter $M=ma$.
%
%
%
% - - - - - - - - - -  fermionic operator   - - - - - 
%
%
%
%
%
\subsubsection{Wilson term and fermionic operator}
Before discretizing the fermionic operator we have to think about the doubling problem arising through naively discretized first derivatives, discussed in \autoref{sec: ferm_doubling}. The free fermionic operator (evaluated in the bosonic vakuum) represented in a momentum basis reads
%
%
\begingroup
\everymath{\footnotesize}
\begin{flalign}
\!\!\!
K_{\rm F} &= 
\begin{pmatrix}
0 & -p_{0}\mathds{1}_{4} & \left(p_{1}-i\tfrac{m}{2}\right)\rho^{M}u^{M} & 0 \\ 
-p_{0}\mathds{1}_{4} & 0 & 0 & \left(p_{1}-i\tfrac{m}{2}\right)\rho_{M}^{\dagger}u^{M} \\ 
-\left(p_{1}+i\tfrac{m}{2}\right)\rho^{M}u^{M} & 0 & 0 & -p_{0}\mathds{1}_{4} \\ 
0 & -\left(p_{1}+i\tfrac{m}{2}\right)\rho_{M}^{\dagger}u^{M} & -p_{0}\mathds{1}_{4} & 0
\end{pmatrix} ,
\raisetag{-8pt}
\end{flalign}
\endgroup
%
%
and has determinant
%
%
\begin{equation}
\det K_{\rm F} = \left( p_{0}^{2} + p_{1}^{2} + \frac{m^{2}}{4} \right)^{8}.
\label{eq: det_K_F}
\end{equation}
%
%
The fermionic propagators are proportional to the inverse of dynamic operators. It is therefore reasonable that the naive discretization (like in (\ref{eq: naive_first_dev}))
%
%
\begin{equation}
p_{i} \to \mathring{p}_{i} \equiv \frac{1}{a}\sin (p_{i}a)
\end{equation}
%
%
will give rise to doublers. For this reason we would like to introduce a \names{Wilson}-like term that cancels the additional poles in the fermionic propagator. It should obey the following conditions:
%
%
\begin{itemize}
\item preserve the maximal amount of symmetries and relevant matrix properties,
%
\item give the correct continuum limit for $a \to 0$,
%
\item should not give rise to a complex phase.
\end{itemize}
%
%
We saw from (\ref{eq: U1_sym}) that fermions transform according to a specific charge $q$ under $U(1)$ transformations
\begin{equation}
\psi \to e^{iq\alpha} \psi.
\end{equation}
%
%
Therefore all the components of $\mathcal{O}_{\rm F}$ in (\ref{OF}) that are zero are necessary to preserve the $U(1)$ symmetry so that only fields with complementary charges can couple. The structure of the other components is determined by the $SO(6)$ symmetry for which reason there is only a small margin in applying a \names{Wilson} term that obeys all the symmetries. In fact it was not possible to construct such an operator that also preserves relevant matrix properties like skew- and $\Gamma_{5}$-symmetry and also leads to the correct perturbative 1-loop coefficient in (\ref{eq: scaling_fct}). \\
We therefore chose to explicitly break the $U(1)$ symmetry and introduce a \names{Wilson}-like operator on the diagonal blocks of $\mathcal{O}_{\rm F}$. In terms of the free fermion operator in momentum space this takes the form
%
%
\begingroup
\everymath{\footnotesize}
\begin{flalign}
\!\!\!
\hat{K}_{\rm F} &= 
\begin{pmatrix}
W_{+} & -\mathring{p}_{0}\mathds{1}_{4} & \left(\mathring{p}_{1}-i\tfrac{m}{2}\right)\rho^{M}u^{M} & 0 \\ 
-\mathring{p}_{0}\mathds{1}_{4} & -W_{+}^{\dagger} & 0 & \left(\mathring{p}_{1}-i\tfrac{m}{2}\right)\rho_{M}^{\dagger}u^{M} \\ 
-\left(\mathring{p}_{1}+i\tfrac{m}{2}\right)\rho^{M}u^{M} & 0 & W_{-} & -\mathring{p}_{0}\mathds{1}_{4} \\ 
0 & -\left(\mathring{p}_{1}+i\tfrac{m}{2}\right)\rho_{M}^{\dagger}u^{M} & -\mathring{p}_{0}\mathds{1}_{4} & -W_{-}^{\dagger}
\end{pmatrix} ,
\raisetag{-8pt}
\end{flalign}
\endgroup
%
%
where
%
%
\begin{equation}
W_{\pm} = \frac{ra}{2} \left( \hat{p}_{0}^{2} \pm i \hat{p}_{1}^{2} \right) \rho^{M}u^{M}, \qquad \vert r \vert =1,
\end{equation}
%
%
and similar to (\ref{eq: wilson_op})
%
%
\begin{equation}
\hat{p}_{i} \equiv \frac{2}{a} \sin \frac{p_{i}a}{2}\,.
\end{equation}
%
%
This leads to the analogue expression of (\ref{eq: det_K_F})
%
%
\begin{equation}
\det \hat{K}_{\rm F} = \left( \mathring{p}_{0}^{2} + \mathring{p}_{1}^{2} +\frac{r^{2}a^{2}}{4}\left( \hat{p}_{0}^{4}+\hat{p}_{1}^{4} \right) + \frac{m^{2}}{4} \right)^{8}\,.
\end{equation}
%
%
If we substitute this into the denominator of (\ref{eq: 1_loop}) and apply the replacement $p_{i}^{2} \to \hat{p}_{i}^{2}$ in the numerator, the discretized equivalent of the 1-loop free energy can be defined by 
%
%
\begin{equation}
\Gamma^{(1)}_{\rm LAT} = -\ln Z^{(1)}_{\rm LAT} = \mathcal{I}(a).
\end{equation}
%
%
With $r=1$ and momentum integration over the first \names{Brillouine} zone this results in
%
%
\begin{equation}
\mathcal{I}(a)=
\end{equation}