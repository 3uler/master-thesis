\begin{center}
{\sffamily \bfseries\Large Acknowledgement}\\
\end{center}%\hspace*{\fill}\\[1.5cm]
\vspace{1cm}
%
First of all I would like to thank my supervisor Dr.~Valentina Forini for providing me with this topic, for the opportunity to actively participate in her research project and the constructive remarks on my thesis. I am also very grateful for the guidance of Dr.~Björn Leder, his patience and willingness to answer all my questions. Furthermore I would like to thank Dr.~Edoardo Vescovi, Philipp Krah and Josua Faller for a lot of fruitful discussions and constructive remarks concerning the finalisation of my thesis and the whole Quantum Field and String Theory Group for providing a pleasant and creative atmosphere.\\
Last but not least I would also like to thank my family and friends for their constant support throughout the last years of study and especially the recent time of finishing this thesis.\\
%
%
%
\vspace{3cm}
%
\begin{center}
{\sffamily \bfseries\Large Hilfsmittel}\\
\end{center}%\hspace*{\fill}\\[1.5cm]
\vspace{1cm}
Diese Arbeit wurde in \LaTeX{ } unter Verwendung von TexLive-2015 gesetzt. In der Erstellung der Bibliographie wurde BiBTeX mit der von Niklas Beisert verwalteten Stil-Datei \glqq nb.st\grqq{ }verwendet. Grafiken wurden mit Hilfe der Open-Source Programme Blender und Inkscape und numerische Plots mit Hilfe von MATLAB erstellt. Die Implementierung der für die numerischen Berechnungen verwendeten Programme erfolgte im {\sc Fortran 95} Standard und wurde mittels eines $\text{Intel}^{\circledR}$ Fortran Compilers ausführbar gemacht. Für die Erzeugung von Pseudozufallszahlen und das Lösen von Gleichungssystemen wurden die Routinen \texttt{ranlux}, \texttt{cgne} und \texttt{mscg} aus dem Paket \texttt{openQCD} adaptiert. Numerische Fehleranalyse erfolgte unter Zuhilfenahme der MATLAB Routine \texttt{UWerr} aus \cite{Wolff:2003sm}.