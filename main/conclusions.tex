\chapter{Conclusion and outlook}\label{ch: conclusion}
In this thesis we studied a discretisation of the $AdS$-light-cone gauge-fixed action for the Type IIB \names{Green-Schwarz} superstring to investigate the cusp anomalous dimension of $\mathcal{N}=4$ SYM in a regime of finite \names{'t Hooft} coupling from a string theory perspective via the $AdS_{5}/CFT_{4}$ duality. For the convenience of the we reviewed the most essential aspects of superstring theory and $AdS/CFT$ correspondence, and provided a fundamental basis of numerical tools.\\
We measured the (derivative of the) cusp anomaly of planar $\mathcal{N}=4$ SYM as derived from string theory and further the masses of two bosonic excitations of the $AdS$ \names{Lagrangian} propagating transverse to the relevant, classical solution. The derived data is in good agreement with the perturbative regime of the sigma-model (large $g=\sqrt{\lambda}/4\pi$). For smaller values of $g$ our  precision is not high enough and continuum extrapolations do not show the predicted upwards trend, stated by the perturbative prediction for the cusp anomaly. But in turn we can observe quite well a qualitative agreement to the predicted bending down, coming from the integrability of the model.\\
Lattice simulations were performed utilising a rational hybrid Monte Carlo \linebreak(RHMC) algorithm and a \names{Wilson} term (to avoid fermion doublers) which explicitly breaks the $SO(2)$ symmetry of the model. To perform a well defined continuum extrapolation we employ a line of constant physics that demands physical masses to be kept constant while approaching the continuum limit. Therefore no tuning of the bare mass parameter of the theory (the light-cone momentum $p^{+}$ that was later substituted by the mass scale $m$) is required which is confirmed by our measurement of the masses of the bosonic excitations.\\
In measuring the action we observe a divergence proportional to the world volume $(\sim 2L^{2})$ with a coupling dependent coefficient that has to be subtracted by hand. Within lattice regularisation such divergences are expected. In continuum perturbation theory power-divergences arising in this model are set to zero, using dimensional regularisation \cite{Giombi:2009gd}. In a hard-cutoff regularisation approach on the other hand, like the lattice one, the appearance of power-divergences is inevitable and their further handling is a highly non-trivial task. In our case we proceeded by a non-perturbative subtraction of the divergences which works very well, but in general this method should be treated with caution since potential ambiguities can arise, like errors that diverge in the continuum limit.\\
To compare our simulation data with the predictions, coming from integrability (\autoref{fig: f_prime_Lm4}), we employ a matching procedure, based on the hypothesis of a simple scaling relation between the bare couplings on the lattice and in the continuum, respectively. It is possible that this bold assumption is not valid and the relation between the couplings is of a more complex nature. This could be supported by a further investigation of the model for smaller $g$ which is currently prevented by the instability of the simulations for $g <5$.\\
An improvement, that could be achieved compared to a previous status of the project \cite{Bianchi:2016cyv}, is that there is no appearance of a complex phase in the \names{Pfaffian} of the fermionic operator $\hat{\mathcal{O}}_{\rm F}$. We can think of the eigenvalues of $\hat{\mathcal{O}}_{\rm F}$ as continuous functions of the Monte Carlo time along a RHMC trajectory. In order to change the \names{Pfaffian's} sign, some eigenvalues of $\hat{\mathcal{O}}_{\rm F}$ would need to transit through zero. We checked that for our simulations for $g \geq 10$ the smallest eigenvalues occupy a sufficiently large separation from zero to ensure  a stable simulation with positive \names{Pfaffian} and without the requirement to apply a reweighting. We can therefore say that we eliminated a severe sign problem occurring  in \cite{Bianchi:2016cyv} in the region of $g\geq 10$ which already tests the non-perturbative regime of the model. Further we did not find any evidence that the results for our considered observables are influenced by finite volume effects or  the occurrence of fermion doublers. For small $g$ the simulations are troubled with small eigenvalues, large condition numbers and potential sign flips of the \names{Pfaffian} that might be correlated.\\
In the future this problem could be addressed numerically by an improvement of the rational approximation or by the use of a preconditioned conjugate gradient solver, providing a better stability in the regime of small eigenvalues.\\
From a more theoretical point of view one could further try to consider other observables related to the cusp anomaly that do not require a subtraction of divergences, and also measure the masses of fermionic fields in the model to test the validity of the applied line of constant physics.\\
In the end one can say that the study at hand provides results that are in qualitative agreement with the theoretical predictions, but still contain certain peculiarities, but it is, nonetheless, an interesting approach to investigate $AdS/CFT$ in a non-perturbative regime that should be continued.