%-englische-Zusammenfassung---------------------------------------
%
%
\begin{center}
{\sffamily \bfseries\Large Abstract}\\
\end{center}%\hspace*{\fill}\\[1.5cm]
\vspace{1cm}
%
\addcontentsline{toc}{chapter}{Abstract}
%\setcounter{page}{2} % Nach Bedarf anpassen!
This thesis discusses the discretisation of an $AdS$-light-cone gauge-fixed action for the Type IIB \names{Green-Schwarz} superstring. We measure the masses of two bosonic excitations transverse to the classical null-cusp solution of the $AdS$ \names{Lagrangian} and investigate the cusp anomalous dimension of $\mathcal{N}=4$	SYM from a string theory perspective via the $AdS/CFT$ duality. For both observables we find a good agreement in the perturbative regime of the sigma-model at large \names{'t Hooft} coupling. For smaller couplings we observe a quadratic divergence that we treat by non-perturbative subtraction. After taking the continuum limit we find a qualitative agreement with the non-perturbative predictions from the integrability of the model. We eliminated a sign problem induced by the presence of a complex phase in the fermionic determinant by introducing an alternative fermionic linearisation. For small couplings we then face the problem of treating operators with small eigenvalues, not effectively treatable via reweighting.\\
\vspace{1.5cm}

%-deutsche Zusammenfassung----------------------------------------
%
\begin{center}
{\sffamily \bfseries\Large Zusammenfassung}\\
\end{center}%\hspace*{\fill}\\[1.5cm]
\vspace{1cm}
%
%\setcounter{page}{3} % Nach Bedarf anpassen!
Diese Arbeit beschreibt die Diskretisierung der Wirkung des Type IIB \names{Green-Schwarz} Superstrings unter $AdS$-Lichtkegeleichfixierung. Es werden die Massen zweier bosonischer Anregungen transversal zur klassischen Lösung lichtartiger Spitzen (null-cusp) des $AdS$-\names{Lagrangians} bestimmt. Das Hauptaugenmerk liegt bei der Ermittelung der anomalen Dimension lichtartiger Spitzen in $\mathcal{N}=4$ SYM mittels einer stringtheoretischen Betrachtung unter Ausnutzung der \linebreak $AdS/CFT$ Korrespondenz. Für beide Observablen findet sich eine gute Übereinstimmung mit dem perturbativen Bereich bei starker \names{'t Hooft}-Kopplung. Für kleinere Kopplungen treten quadratische Divergenzen auf, die einer nicht-perturbativen Subtraktion unterzogen werden. Nach der Kontinuumsextrapolation zeigt sich eine qualitative Übereinstimmung der Ergebnisse mit den nicht-perturbativen Vorhersagen, welche aus der Integrabilität dees zugrundeliegenden Modells gewonnen wurden. Es konnte erfolgreich in Vorzeichenproblem mittels einer alternativen Linearisierung der Fermionen eliminiert werden. Für kleine Kopplungen werden jedoch Operatoren mit kleinen Eigenwerten zum Problem, welche sich nicht effizient mit der Methode des \glqq reweighting\grqq{ }behandeln lassen.
% hier werden die deutsche Schlagwörter aus Metadaten übernommen
%\dckeywordsde