%-Eingabe der Metadaten des Titelblattes--------------------------

%-Daten des Autors / Authors Data---------------------------------

\newcommand{\dcauthorpre}{} 
\newcommand{\dcauthorsurname}{Töpfer} 
\newcommand{\dcauthorname}{Philipp} 
\newcommand{\dcauthoradd}{30.07.1990, Apolda}

%-Titel und Untertitel / Title and subtitle-----------------------

\newcommand{\dctitle}{\LARGE\textsc{Lattice discretization of the Green-Schwarz superstring}} 
\newcommand{\dcsubtitle}{~}  
% Falls dcsubtitle NICHT verwendet werden soll, {\dcsubtitle}{~} eingeben.

%-Eingabe der Betreuuernahmen / Names of the consultants---------

\newcommand{\dcconsulta}{Prof. Dr. Valentina Forini} 
\newcommand{\dcconsultb}{Dr.  Björn Leder}

%-Eingabe der Gutachternamen / Names of the approvals-------------

\newcommand{\dcapprovala}{Prof. Dr. Valentina Forini} 
\newcommand{\dcapprovalb}{Dr.  Björn Leder} 
\newcommand{\dcapprovalc}{} 

%-Information zur Universitaet------------------------------------

\newcommand{\dcdegree}{Master of Science (M. Sc.)} 
\newcommand{\dcsubject}{Physik} 
\newcommand{\dcfaculty}{Mathematisch-Naturwissenschaftlichen Fakultät I}
\newcommand{\dcinstitute}{Institut für Physik}
\newcommand{\dcuniversity}{Humboldt-Universität zu Berlin}
\newcommand{\dcdean}{}
\newcommand{\dcpresident}{}

%-Pruefungsdaten: eingereicht und mdl. Pruefung-------------------
%-data of submission and oral exam--------------------------------

\newcommand{\dcdatesubmitted}{20. Dezember 2016} %auch wenn nicht auf dem Titelblatt, bitte erf�llen!
\newcommand{\dcdateexam}{} 

%-deutsche Schlagwoerter / german keywords------------------------

\newcommand{\dckeydea}{Schlagwort 1}
\newcommand{\dckeydeb}{Schlagwort 2}
\newcommand{\dckeydec}{Schlagwort 3}
\newcommand{\dckeyded}{Schlagwort 4}

% Folgende Zeile bitte nicht aendern!
\newcommand{\dckeywordsde}{\vfill \raggedright {\textbf{Schlagw\"orter:}}\\ \dckeydea, \dckeydeb, \dckeydec, \dckeyded \\}

%-englische Schlagwoerter / english keywords----------------------

\newcommand{\dckeyena}{keyword 1}
\newcommand{\dckeyenb}{keyword 2}
\newcommand{\dckeyenc}{keyword 3}
\newcommand{\dckeyend}{keyword 4}

% Folgende Zeile bitte nicht aendern!
\newcommand{\dckeywordsen}{\vfill \raggedright {\textbf{Keywords:}}\\ \dckeyena, \dckeyenb, \dckeyenc, \dckeyend \\}

\newcommand{\dcpdfsubject}{Dissertation}