% - - - - - - - - - - Appendices - - - - - - - - - - - - - -
%
%
%- - - - - - - -  Grassmann numbers  - - - - - - - - - -
%
%
%
\section{Grassmann Numbers}
../grassmann_algebra/grassmann_alg.tex
%
%
% - - - - - - -  AdS spacetime - - - - - - - - - - - - -
%
%
%
\section[$AdS_{5}\times S^{5}$ spacetime]{$\mathitbf{AdS_{5}\times S^{5}}$ spacetime}
The $AdS_{5}\times S^{5}$ space is in the central focus of the gauge/gravity duality and should be concerned more deeply. It is a direct product of five dimensional Anti-de Sitter ($AdS$) space and a five dimensional compact sphere. Both are maximally symmetric spaces and therefore inherit the isometry groups $SO(2,4)$ in case of $AdS_{5}$ and respectively $SO(6)$ for $S^{5}$ \cite{Ammon:2015wua}. This is an important fact for the $AdS/CFT$ correspondence since the direct product of these groups has the same amount of degrees of freedom as the superconformal group $SO(2,4)\times SO(6) = SU(2,2\vert 4)$ as the undelying symmetry group of $\mathcal{N}=4$ super \names{Yang-Mills} theory in four dimensional \names{Minkowski} space.
%
%
\subsection[$AdS_{5}$ space]{$\mathitbf{AdS_{5}}$ space}\label{AdS5}
Since the construction of a sphere is rather simple, we focus on the Anti-de Sitter space. $AdS_{5}$ is a hyperboloid with constant negative curvature, that can be embedded in six dimensional \names{Minkowski} spacetime $X = (X^{0},X^{1},\ldots,X^{5}) \in \mathbb{R}^{4,2}$, with metric ${\tilde{\eta} = {\rm diag}(-,+,+,+,+,-)}$, so that
\begin{equation}
\dd s^{2} = -\left(\dd X^{0}\right)^{2} + \left(\dd X^{1}\right)^{2} + \cdots + \left(\dd X^{4}\right)^{2} - \left(\dd X^{5}\right)^{2} = \tilde{\eta}_{MN} \dd X^{M}\dd X^{N},
\end{equation}
where $M,N \in {0,\ldots,5}$. $AdS_{5}$ is then given by the hypersurface
\begin{equation}
\tilde{\eta}_{MN}X^{M}X^{N} = -\left(X^{0}\right)^{2} + \left(X^{1}\right)^{2} + \cdots + \left(X^{4}\right)^{2} - \left(X^{5}\right)^{2} = -R^{2},
\label{hyperbol}
\end{equation}
%
%
\begin{figure}[ht!]
\begin{center}
\includegraphics[width=0.6\textwidth]{AdS2.pdf}
\caption{Embedding of $AdS_{2}$ in $\mathbb{R}^{3}$ as a hypersurface given by the equation $-(X_{0})^{2}+(X_{1})^{2}-(X_{3})^{2}=-1$. \label{fig: AdS2}}
\end{center}
\end{figure}
%
%
where $R$ is the radius of curvature of the $AdS_{5}$ space, see also \autoref{fig: AdS2}  for an embedding of $AdS_{2}$ in $\mathbb{R}^{3}$. For large $X^{M}$ the hyperboloid approaches the light-cone of the \names{Minkowski} space $\mathbb{R}^{4,2}$, given by
\begin{equation}
\tilde{\eta}_{MN}X^{M}X^{N} = 0.
\label{M_l_c}
\end{equation}
We therefore can define the `boundary' $\del AdS_{5}$ of Anti-de Sitter space by the set of all lines on the light-cone (\ref{M_l_c}) originating from $0 \in \mathbb{R}^{4,2}$. For a point $X \neq 0$ in $AdS_{5}$ close to the boundary and therefore satisfying (\ref{M_l_c}) we can define $(u,v)$ by
\begin{equation}
u = X^{5}+X^{4}, \qquad v = X^{5}-X^{4},
\end{equation}
so we can rewrite (\ref{M_l_c}) as
\begin{equation}
uv = \eta_{\mu \nu} X^{\mu}X^{\nu},
\end{equation}
with $\mu,\nu \in {0,1,2,3}$ and $\left(\eta_{\mu \nu}\right) = {\rm diag}(-,+,+,+).$ Whenever $v \neq 0$ we can rescale the coordinates to set $v = 1$ and solve for $u$. Therefore one is left with a four dimensional \names{Minkowski} space $\mathbb{R}^{3,1}$. Points with $v=0$ are ``points at infinity'' added to four dimensional \names{Minkowski} space. This makes $\del AdS_{5}$ a conformal compactification of four dimensional \names{Minkowski} space. According to Maldacena \cite{maldacena1} the correspondence between a $\mathcal{N}=4$ theory on $\mathbb{R}^{3,1}$ and Type IIB on $AdS_{5}\times S^{5}$ therefore expresses a string theory on $AdS_{5}\times S^{5}$ in terms of a theory on the boundary and thus is referred to as ``holographic'' \cite{Witten:1998qj}.
%
%
\subsection{Poincaré patch}\label{p_patch}
Let us now introduce a parametrisation of the hyperboloid (\ref{hyperbol}) by the following coordinates $x^{\mu} \in \mathbb{R}$, for $\mu \in {0,1,2,3}$ and $z \in \mathbb{R}_{+}$. The parametrisation in these coordinates is given by (see e.g. \cite{Ammon:2015wua,Bayona:2005nq})
\begin{align}
X^{0} &= \frac{z}{2} \left( 1+ \frac{1}{z^{2}}\left( x_{\mu}x^{\mu} + R^{2}\right) \right), \notag \\
X^{i} &= \frac{R}{z}x^{i},\quad i\in {1,2,3}, \\
X^{4} &= \frac{z}{2} \left( 1 + \frac{1}{z^{2}} \left(x_{\mu}x^{\mu} -R^{2} \right) \right), \notag \\
X^{5} &= \frac{R}{z}x^{0}, \notag
\end{align}
with $x_{\mu}x^{\mu}=\eta_{\mu\nu}x^{\mu}x^{\nu}$ and $\left(\eta_{\mu \nu}\right) = {\rm diag}(-,+,+,+)$. These local coordinates are called \names{Poincaré} patch. The metric of $AdS_{5}$ in the \names{Poincaré} patch reads
\begin{equation}
\dd s^{2} = \frac{R^{2}}{z^{2}}\left(\dd z^{2} + \eta_{\mu \nu} \dd x^{\mu} \dd x^{\nu} \right).
\end{equation}
%
%
For the whole $\AdSS$ space in \names{Poincaré} patch we find 
%
%
\begin{equation}
\dd s^{2} = \frac{R^{2}}{z^{2}}\left(\dd z^{2} + \eta_{\mu \nu} \dd x^{\mu} \dd x^{\nu} \right) + \dd u^{I}\dd u^{I},
\end{equation}
%
%
where $u^{I}$ $(I=1,\ldots,6)$ are coordinates on $S^{5}$ that satisfy $u^{I}u^{I}=R^{2}$. For $z\to 0$ we approach the boundary of $\AdSS$ and we can see from the metric that the contribution of the sphere becomes neglectable close to the boundary. If we include the infinite regime as $\del (\AdSS)$ where $z=0$ we can convince ourselves that the metric is conformally equivalent to 4d \names{Minkowski} space.
%
%
%
%- - - - - - - - - rho matrix conventions - - - - - - - - - - - - -
%
%
%
\section{SO(6) matrices} \label{sec: SO6}
The matrices $\rho^{M}_{ij}$ appearing in the action (\ref{eq: cusp_action}) are the off-diagonal blocks of $SO(6)$ \names{Dirac} matrices $\gamma^{M}$ in chiral representation\footnote{The upper or lower placement of the index $M$ on the block matrices has no meaning and is only changed for the purpose of readability.}
%
%
\begin{equation}
\gamma^{M} \equiv \begin{pmatrix}
0 & \rho_{M}^{\dagger} \\ 
\rho^{M} & 0
\end{pmatrix} 
= \begin{pmatrix}
0 & (\rho^{M})^{ij} \\ 
(\rho^{M})_{ij} & 0
\end{pmatrix} .
\end{equation}
%
%
The $\rho^{M}_{ij}$ shall all skew symmetric and we define the ones with the upper indices to satisfy $(\rho^{M})^{ij}\equiv (\rho^{M}_{ij})^{\dagger}$. We can therefore state the following properties
%
%
\begin{equation}
\rho^{M}_{ij}= - \rho^{M}_{ji}, \qquad  (\rho^{M})^{ij} = - (\rho ^{M}_{ij})^{*},
\end{equation}
%
%
and we chose to use the following representation
%
%
\begin{align}
\rho^{1}_{ij} &= \begin{pmatrix}
0 & 1 & 0 & 0 \\ 
-1 & 0 & 0 & 0 \\ 
0 & 0 & 0 & 1 \\ 
0 & 0 & -1 & 0
\end{pmatrix} , &
%
\rho^{2}_{ij} &= \begin{pmatrix}
0 & i & 0 & 0 \\ 
-i & 0 & 0 & 0 \\ 
0 & 0 & 0 & -i \\ 
0 & 0 & i & 0
\end{pmatrix} , &
%
\rho^{3}_{ij} &= \begin{pmatrix}
0 & 0 & 0 & 1 \\ 
0 & 0 & 1 & 0 \\ 
0 & -1 & 0 & 0 \\ 
-1 & 0 & 0 & 0
\end{pmatrix} , \notag\\[0.6cm]
%
%
%
\rho^{4}_{ij} &= \begin{pmatrix}
0 & 0 & 0 & -i \\ 
0 & 0 & i & 0 \\ 
0 & -i & 0 & 0 \\ 
i & 0 & 0 & 0
\end{pmatrix} , &
%
\rho^{5}_{ij} &= \begin{pmatrix}
0 & 0 & i & 0 \\ 
0 & 0 & 0 & i \\ 
-i & 0 & 0 & 0 \\ 
0 & -i & 0 & 0
\end{pmatrix} , &
%
\rho^{6}_{ij} &= \begin{pmatrix}
0 & 0 & 1 & 0 \\ 
0 & 0 & 0 & -1 \\ 
-1 & 0 & 0 & 0 \\ 
0 & 1 & 0 & 0
\end{pmatrix} . \raisetag{-6pt}
\end{align}
%
%
From the \names{Clifford} algebra of the \names{Dirac} matrices $\lbrace \gamma^{M}, \gamma^{N}\rbrace = 2\delta^{MN}\mathds{1}_{8}$ we can derive the relation
%
%
\begin{equation}
(\rho^{M})^{il} (\rho^{N})_{lj} + (\rho^{N})^{il}(\rho^{M})_{lj} = 2\delta^{MN}\delta^{i}_{j}.
\end{equation}
%
%
The generators of the $SO(6)$ can be built by
%
%
\begin{equation}
{\left(\rho^{MN}\right)^{i}}_{j} = \frac{1}{2} \left[ \left(\rho^{M}\right)^{il} \left(\rho^{N}\right)_{lj} - \left(\rho^{N}\right)^{il} \left(\rho^{M}\right)_{lj} \right].
\end{equation}
%
%
Further relations and identities are 
%
%
\begin{align}
{\left(\rho^{MN}\right)^{i}}_{j} &= {\left(\rho^{MN}\right)^{*}_{i}}^{j}\;, & {\left(\rho^{MN}\right)^{i}}_{j} &= {\left(\rho^{NM}\right)_{j}}^{i}\;,
\label{eq: SO6_id} \\
 (\rho^M)^{im} (\rho^M)^{kn}&=2\epsilon^{imkn}\;, &  (\rho^M)^{im} (\rho^M)_{nj}&=2\left(\delta^i_j \delta^m_n -\delta^i_n \delta^m_j\right)\;,
\end{align}
%
%
\begin{align}
 &\quad \epsilon^{imkn} (\rho^M)_{mj}(\rho^L)_{nl}+\epsilon_{mjnl}(\rho^M)^{im}(\rho^L)^{kn}\\
 &=(\rho^{\{M})^{ik}(\rho^{L\}})^{jl}+\delta^k_j(\rho^L)^{im}(\rho^M)_{ml}+\delta^i_l(\rho^M)^{km}(\rho^L)_{mj}\nonumber \\
 &\quad+\delta^{ML}\left(-4\delta^i_l \delta^k_j +2\delta^i_j\delta^k_l\right)\;,\nonumber\\[0.4cm]
& -{(\rho^{MN})^i}_j {(\rho^{ML})^k}_l n_N n_L=-2(\rho^{N})^{ik} (\rho^{L})_{jl} n_N n_L-\delta^i_j \delta^k_l+2\delta^i_l \delta^k_j \;.
\end{align}
%
%
%
%
%- - - - - - - - - Discrete Fourier Transform - - - - - - - - - -
%
%
%
\section{Discrete Fourier Transform}
\label{app: disc_ft}
To perform a Fourier transform on the lattice one needs to discretize it to deal with a finite sequence of $N$ complex numbers $x_{0},x_{1},\ldots,x_{N-1}$. We define the discrete Fourier transform $X_{k}$ to be a vector in the base of roots of unit with components $x_{n}$ as follows:
\begin{equation}
X_{k}=\dfrac{1}{\sqrt{N}} \sum\limits_{n=0}^{N-1} x_{n} \, e ^{-2\pi  i kn /N} \qquad  k \in \mathbb{Z}.
\label{FT}
\end{equation}
We can limit the domain of $k$ to a finite set, because the exponential is periodic in $k$. In the following we want to stick to the domain $k \in \left[ -\tfrac{N}{2}+1,\ldots,\tfrac{N}{2} \right]$ and restrict us to even $N$. The inverse transform can be defined to be
%
\begin{equation}
 x_{n}= \dfrac{1}{\sqrt{N}} \sum\limits_{k=-N/2+1}^{N/2} X_{k} \, e ^{2 \pi  i nk /N},
 \label{invFT}
 \end{equation}
 where due to periodicy $n$ is in the domain $\left[0,\ldots,N-1\right]$ like defined in the beginning. If we now insert (\ref{FT}) into (\ref{invFT}) we end up with
\begin{align}
x_{n} &=\dfrac{1}{N}\sum\limits_{k=-N/2+1}^{N/2} \left( \sum\limits_{m=0}^{N-1} x_{m}  e ^{-2\pi  i km /N} \right)  e ^{2\pi  i kn /N} \\
%
%
&= \dfrac{1}{N} \sum\limits_{k',m=0}^{N-1} x_{m}  e ^{2\pi  i k'(n-m) /N},
\end{align}
where we shifted the summation for $k'$ to be in the same domain as $m$. This is now only equal to $x_{n}$ if the following important relation holds true
%
%
\begin{equation}
\dfrac{1}{N} \sum\limits_{k'=0}^{N-1}  e ^{2\pi  i k'(n-m) /N} = \delta_{m,n}.
\end{equation}
We can prove this easily. For $n=m$ this is trivial and for $n\neq m$ we make use of the geometric series
\begin{equation}
\sum\limits_{k'=0}^{N-1} \left( e ^{2\pi i  c/N }\right)^{k'} = \dfrac{1- e ^{2\pi i  c}}{1- e ^{2\pi i  c/N }} = 0, \qquad  \vert c\vert = \vert n-m \vert \in [1,\ldots,N-1].
\end{equation}
%
%
Now we want to apply the Fourier transform on the two dimensional lattice used throughout this thesis. We defined the lattice to be
\begin{equation}
\mathit{\Lambda}=\left\lbrace  n=\left(n_{0},n_{1}\right) \vert n_{0}=1,\ldots,N_{0}\,; \; n_{1}=1,\ldots,N_{1} \right\rbrace,
\end{equation}
with $N_{0}=T$ and $N_{1}=L$. Therefore the total number of lattice points is given by
\begin{equation}
\vert \mathit{\Lambda}\vert \equiv TL.
\end{equation}
Following \cite{gattringer2009quantum} we now want to calculate the discrete Fourier transform $\tilde{f}(p)$ of a function $f(n)$. Here for $f(n)$ we impose toroidal boundary conditions
\begin{equation}
f(n+\hat{\imath}N_{i})= e ^{2\pi i \theta_{i}}f(n),
\label{eq: boundary_cond}
\end{equation}
where $\hat{\imath}$ is a unit vector in the $i$-direction and $\theta_{i}=0$ corresponds to periodic and $\theta_{i}=1/2$ to anti-periodic boundary conditions. The discrete momentum space corresponding to these boundary conditions is given by
\begin{equation}
\mathit{\widetilde{\Lambda\,}}= \left\lbrace p=(p_{0},p_{1}) \vert\, p_{i}=\dfrac{2\pi}{aN_{i}}(k_{i}+\theta_{i}) ,\, k_{i}=-\dfrac{N_{i}}{2}+1,\ldots,\dfrac{N_{i}}{2} \right\rbrace.
\end{equation}
With (\ref{FT}) and (\ref{invFT}) we can express the the Fourier transform as
\begin{equation}
\tilde{f}(p)=\dfrac{1}{\sqrt{\vert\mathit{\Lambda}\vert}} \sum\limits_{n\in \mathit{\Lambda}} f(n)\,  e ^{- i  p\cdot na}
\end{equation}
and for the inverse transform we find
\begin{equation}
f(n) = \dfrac{1}{\sqrt{\vert\mathit{\Lambda}\vert}} \sum\limits_{p\in \widetilde{\mathit{\Lambda}\,}} \tilde{f}(p)\,  e ^{ i p\cdot na}.
\end{equation}
Here again the important relations hold
\begin{align}
\dfrac{1}{\vert\mathit{\Lambda}\vert} \sum\limits_{n\in \mathit{\Lambda}} \text{exp}( i (p-p')\cdot na) &=\delta(p-p') \equiv \delta_{k_{0},k'_{0}}\delta_{k_{1},k'_{1}} \\
\dfrac{1}{\vert\mathit{\Lambda}\vert} \sum\limits_{p\in  \widetilde{\mathit{\Lambda}\,}} \text{exp}( i p\cdot (n-n')a) &=\delta(n-n') \equiv \delta_{n_{0},n'_{0}}\delta_{n_{1},n'_{1}}.
\end{align}
%
%
%
%
%
%
% - - - -  -- - - - - - - - -   discretized fermion matrix  - - - - - - - - - - - - - - - -
%
%
%
%
%
\section{Discretized fermion matrix}
\label{app: disc_ferm}
Here we want to present the details leading to the discretized version (\ref{eq: disc_OF}) of the operator $\mathcal{O}_{\rm F}$. As emphasized in section \ref{sec: ferm_op} $\hat{\mathcal{O}}_{\rm F}$ is a $16V\times 16V$ matrix and we will continue to use the notation introduced there to write subdivided blocks of size $4V\times 4V$ to result in
%
%
\begingroup
\everymath{\footnotesize}
\begin{flalign}
\!
\hat{\mathcal{O}}_{\rm F} = 
\begin{pmatrix}
\hat{W}_{+} & i\bar{\Delta}^{\rm A}_{0} & -i\left(\vec{\Delta}^{Z}_{1} + \tfrac{M}{2}\bar{Z}\right) & 0 \\ 
i\bar{\Delta}^{\rm A}_{0} & -\hat{W}_{+}^{\dagger} & 0 & -i\left(\vec{\Delta}^{Z^{\dagger}}_{1} + \tfrac{M}{2}\bar{Z}^{\dagger}\right)  \\ 
i\left(\cev{\Delta}^{Z}_{1} - \tfrac{M}{2}\bar{Z}\right)  & 0 & \!\!\!\!\!\!\!\!\! 2\left(\bar{\Delta}^{x}_{1}-\tfrac{M}{2}\bar{Z}^{x}\right)+\hat{W}_{-} & i\bar{\Delta}^{\rm A}_{0} -\hat{A}^{\rm T} \\ 
0 & i\left(\cev{\Delta}^{Z^{\dagger}}_{1} - \tfrac{M}{2}\bar{Z}^{\dagger}\right) & i\bar{\Delta}^{\rm A}_{0} +\hat{A} & \!\!\!\!\! 2\left(\bar{\Delta}^{x^{*}}_{1}-\tfrac{M}{2}\bar{Z}^{x^{*}}\right)-\hat{W}_{-}^{\dagger}
\end{pmatrix} .
\raisetag{-8pt}
\end{flalign}
\endgroup
%
%
First of all we should mention that the operator $\mathcal{O}_{\rm F}$ was of dimension $[a]^{-1}$ which has been absorbed into the fermionic fields to make them dimensionless. Therefore $\hat{\mathcal{O}}_{\rm F}$ is dimensionless as well, as are now all the bosonic and fermionic fields. Further $\hat{\mathcal{O}}_{\rm F}$ needs to be anti-symmetric which is why all the discretized derivatives need to be symmetric finite differences. As it is standard in lattice QCD to use anti-periodic boundary conditions in the temporal direction for the fermions \cite{montvay_lattice}, we will apply those here as well. For the spacial direction periodic boundary conditions shall be used. So starting from block $(1,2)$ we make the transition
%
%
\begin{equation}
i \del_{t}\mathds{1}_{4} \longrightarrow i \bar{\Delta}^{\rm A}_{0},
\end{equation}
%
%
where we defined
%
%
\begin{align}
\bar{\Delta}^{\rm A}_{0} &\equiv \bar{\Delta}^{\rm a}_{0} \otimes \mathds{1}_{4}, &
\bar{\Delta}^{\rm a}_{0} &\equiv \left(\bar{\Delta}^{\rm a}_{0}\right)(l,p) = \frac{1}{2}\left( \delta_{l_{+\hat{1}},p}^{\rm a}-\delta_{l_{-\hat{1}},p}^{\rm a}\right).
\end{align}
%
%
Thereby the superscripts \textit{A, a} refer to the property of the finite differences to respect anti-periodic boundary conditions, whereas \textit{p} will refer to periodic ones. In the next block $(1,3)$ we have
%
%
\begin{equation}
-i \rho^{M}\left(\del_{s} +\tfrac{m}{2}\right) \frac{z^{M}}{z^{3}} \longrightarrow -i \left( \vec{\Delta}_{1}^{Z} +\tfrac{M}{2} \bar{Z}\right).
\end{equation}
%
%
Since $z^{M}/z^{3}$ is on the right, the derivative will also act on this term which we will have to consider in the discretization. Thus we introduce the definitions
%
%
\begin{gather}
Z \equiv Z_{ij}(l) = \rho_{ij}^{M} \frac{z^{M}(l)}{z^{3}(l)}, \qquad \bar{Z} \equiv \bar{Z}_{ij}(l,p) = \delta_{l,p} Z_{ij}(l), \\
\vec{\Delta}_{1}^{Z} \equiv \left(\vec{\Delta}_{1}^{Z}\right)_{ij}(l,p) = \frac{1}{2} \left[ \delta_{l_{+\hat{1}},p}^{\rm p}Z_{ij}(l_{+\hat{1}})-\delta_{l_{-\hat{1}},p}^{\rm p} Z_{ij}(l_{-\hat{1}})\right].
\end{gather}
%
%
The right arrow shall indicate that $Z$ is on the right and respected by the derivative. In block $(3,1)$ we also have the same operator with arrow to the left which indicates that the derivative will not act on $Z$
%
%
\begin{equation}
\cev{\Delta}_{1}^{Z} \equiv \left(\cev{\Delta}_{1}^{Z}\right)_{ij}(l,p) = \frac{1}{2} \left[ \delta_{l_{+\hat{1}},p}^{\rm p}-\delta_{l_{-\hat{1}},p}^{\rm p}\right] Z_{ij}(l) .
\end{equation}
%
%
In block $(3,3)$ we observe
%
%
\begin{equation}
2\frac{z^{M}}{z^{4}}\rho^{M}\left(\del_{s}x-\tfrac{m}{2}x\right) \longrightarrow  2\left( \bar{\Delta}_{1}^{x}-\tfrac{M}{2}\bar{Z}^{x} \right),
\end{equation}
%
%
where
%
%
\begin{gather}
\bar{Z}^{x} \equiv \bar{Z}^{x}_{ij}(l,p) = \delta_{l,p}\frac{Z_{ij}(l)}{z(l)}x(l), \\
\bar{\Delta}_{1}^{x} \equiv \left(\bar{\Delta}_{1}^{x}\right)_{ij}(l,p) = \frac{1}{2} \frac{Z_{ij}(l)}{z(l)} \left[ \delta_{l_{+\hat{1}},p}^{\rm p}x(l_{+\hat{1}})-\delta_{l_{-\hat{1}},p}^{\rm p} x(l_{-\hat{1}})\right].
\end{gather}