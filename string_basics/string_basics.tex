\section{Basics on superstring theory}
In this chapter we want to give a brief introduction to the basic concepts of string theory, starting from its simplest possible form, the bosonic string and further develop the subject to a supersymmetric version including fermions. Since we want to apply the concepts of $AdS/CFT$, we are mostly interested in strings propagating in $AdS$ backgrounds. The following assertions are mainly based in the remarks in \cite{Polchinski:1998rq,Polchinski:1998rr}.
%
%
%
% - - - - - - - -  bosonic string  - - - - - - - -
%
%
%
%
\subsection{The bosonic string}
The basic idea of string theory is that the fundamental objects described by the theory are extended from point-like particles to one-dimensional strings propagating through spacetime and sweeping out a $(1+1)$-dimensional surface $\widetilde{\Sigma}$ called the string worldsheet. The string can be parameterized by two coordinates, the proper time $\tau$ and the spatial extent $\sigma$ of the string which we will also denote by $(\sigma^{0},\sigma^{1})=(\tau,\sigma)$. We can then apply a homeomorphism $\big(\Sigma,\phi)$ that maps the 2d sheet $\Sigma$ into the target space which is a $D$-dimensional \names{Minkowski} space\footnote{We will utilize the metric $\eta_{MN}=\text{diag}(-,+,+,\ldots,+) with M,N=0,1,2,\ldots,D-1.$} $\mathbb{R}^{D-1,1}$ with embedding coordinates $X^{\mu}(\tau,\sigma)$ and therefore
%
%
\begin{equation}
\begin{alignedat}{3}
\phi :\quad  \Sigma = (\tau_{\rm in},\tau_{\rm fin}) \times (0,\sigma) \quad &\longrightarrow \quad \widetilde{\Sigma} \\
(\tau,\sigma) \qquad\quad &\longrightarrow \quad X^{\mu}(\tau,\sigma). 
\end{alignedat}
\end{equation}
%
%
The simplest possible action is proportional to the area of the surface of $\widetilde{\Sigma}$, given by
%
%
\begin{equation}
S_{\rm NG} = -\frac{1}{2\pi\alpha'} \int\limits_{\Sigma} \dd^{2}\sigma \, \sqrt{-\gamma},
\end{equation}
%
%
where $\gamma$ is the \names{Gramian} determinant of the embedding
%
%
\begin{align}
\gamma &= \det \big(\gamma_{\alpha\beta}\big), & \gamma_{\alpha\beta} &= \del_{\alpha}X^{M}\del_{\beta}X^{N}\eta_{MN},
\end{align}
%
%
where the induced metric $\gamma_{\alpha\beta}$ is the pull-back of the flat \names{Minkowski} metric $\eta_{MN}$. We also defined $\dd^{2}\sigma = \dd\sigma^{0}\dd\sigma^{1}$ and the parameter in front of the integral is the inverse string tension with $\alpha'$ being related to the string length $l_{\rm s}$ by $\alpha'=l_{\rm s}^{2}$. Due to the square root this so called \names{Nambu-Goto} action is not very suitable for a path integral quantization. To get rid of the square root one can introduce an auxiliary metric $h_{\alpha\beta}(\sigma)$ with $h_{\alpha\beta}h^{\beta\rho}=\delta_{\alpha}^{\rho}$ and define the \names{Polyakov} action by
%
%
\begin{equation}
S_{\rm P} = -\frac{1}{4\pi\alpha'} \int\limits_{\Sigma}\dd^{2}\sigma\, \sqrt{-h}h^{\alpha\beta}\gamma_{\alpha\beta}.
\end{equation}
%
%
This action is equivalent to the \names{Nambu-Goto} action on the classical level which can be proved by deriving the equations of motion for $h_{\alpha\beta}$, $\delta S/ \delta h^{\alpha\beta} = 0$. We can use this to define the corresponding energy-momentum tensor
%
%
\begin{equation}
T_{\alpha\beta} \equiv -4\pi\alpha' \frac{1}{\sqrt{-h}} \frac{\delta S}{\delta h^{\alpha\beta}} = \gamma_{\alpha\beta} - \frac{1}{2} h_{\alpha\beta} h^{\rho\sigma}\gamma_{\rho\sigma} = 0.
\end{equation}
%
%
From the constraint that $T_{\alpha\beta}$ has to vanish we can derive the classical equivalence of the actions $S_{\rm NG}$ and $S_{\rm P}$ with
%
%
\begin{equation}
\sqrt{-\gamma} = \frac{1}{2} \sqrt{-h} h^{\rho\sigma} \gamma_{\rho\sigma}.
\end{equation}
%
%
We will thus use the the \names{Polyakov} action in the following, since it is easier to handle. But before we continue, we want to analyse the symmetries preserved by the \names{Polyakov} action.
%
%
\begin{itemize}
\item\textit{Global D-dimensional Poincarè invariance} 
%
\begin{equation}
X^{M} \rightarrow \widetilde{X}^{M} = \Lambda_{N}^{M} X^{N} + a^{M}, \qquad \delta	h_{\alpha\beta} = 0,
\end{equation}
%
with $\Lambda_{N}^{M}$ and $a^{M}$ being $D$-dimensional \names{Lorentz} transformations and spacetime translations.
%
%
\item\textit{Reparametrisation invariance of the worldsheet (or diffeomorphism invariance)}\\
We can choose a reparametrisation of the worldsheet coordinates $\sigma^{\alpha}\rightarrow \tilde{\sigma}^{\alpha}(\sigma)$, where the fields $X^{M}$ and the 2d metric transform according
%
\begin{equation}
\begin{alignedat}{3}
X^{M}(\sigma) &\rightarrow \widetilde{X}^{M}(\tilde{\sigma}) = X^{M}(\sigma), \\
h_{\alpha\beta}(\sigma) &\rightarrow \tilde{h}_{\alpha\beta}(\tilde{\sigma}) = \frac{\del \sigma^{\alpha}}{\del \tilde{\sigma}^{\gamma}} \frac{\del \sigma^{\beta}}{\del \tilde{\sigma}^{\delta}}h_{\gamma\delta}(\sigma).
\end{alignedat}
\end{equation}
%
This is a gauge symmetry on the worldsheet.
%
%
\item\textit{Weyl invariance} 
\begin{align}
\widetilde{X}^{M}(\sigma) &= X^{M}(\sigma), &  \tilde{h}_{\alpha\beta}(\sigma)&=\Omega^{2}h_{\alpha\beta}(\sigma).
\end{align}
\end{itemize}