\chapter{Basics on superstring theory}\label{ch: string_basics}
In this chapter we want to give a brief introduction to the basic concepts of string theory, starting from its simplest possible form, the bosonic string, and further develop the subject to a supersymmetric version including fermions. Since we will work in the framework of the $AdS/CFT$ correspondence (see \autoref{sec: preleminaries}), we are mostly interested in strings propagating in $AdS$ backgrounds. The following sections are mainly based on \cite{Polchinski:1998rq,Polchinski:1998rr,Ammon:2015wua,Becker:2007zj}.
%
%
%
% - - - - - - - -  bosonic string  - - - - - - - -
%
%
%
%
\section{The bosonic string}
The basic idea of string theory is that the fundamental objects described by the model are one-dimensional strings, propagating through spacetime and sweeping out a $(1+1)$-dimensional surface $\widetilde{\Sigma}$ called the string worldsheet. The string can be parameterised by two coordinates, the proper time $\tau$ and the spatial extent $\sigma$ of the string which we will also denote by $(\sigma^{0},\sigma^{1})=(\tau,\sigma)$. We can then apply a homeomorphism $\varphi$ that maps the 2d sheet $\Sigma$ into the target space, e.g. $D$-dimensional \names{Minkowski} space\footnote{We will use the metric $\eta_{MN}=\text{diag}(-,+,+,\ldots,+)$ with $M,N=0,1,2,\ldots,D-1.$} $\mathbb{R}^{D-1,1}$ with embedding coordinates $X^{M}(\tau,\sigma)$
%
%
\begin{equation}
\begin{alignedat}{3}
\varphi :\quad  \Sigma = (\tau_{\rm in},\tau_{\rm fin}) \times (0,\sigma_{0}) \quad &\longrightarrow \quad \widetilde{\Sigma} \\
(\tau,\sigma) \qquad\quad &\longrightarrow \quad X^{M}(\tau,\sigma).
\end{alignedat}
\end{equation}
%
%
The simplest possible action of the string, the \names{Nambu-Goto} action, is proportional to the area of the surface $\widetilde{\Sigma}$ and is given by
%
%
\begin{equation}
S_{\rm NG} = -\frac{1}{2\pi\alpha'} \int\limits_{\Sigma} \dd^{2}\sigma \, \sqrt{-\gamma},
\end{equation}
%
%
where $\gamma$ is the \names{Gramian} determinant of the embedding
%
%
\begin{align}
\gamma &= \det \big(\gamma_{\alpha\beta}\big), & \gamma_{\alpha\beta} &= \del_{\alpha}X^{M}\del_{\beta}X^{N}\eta_{MN}, & \del_{\alpha}&\equiv \frac{\del}{\del\sigma^{\alpha}},\quad \alpha=0,1\;,
\end{align}
%
%
and the induced metric $\gamma_{\alpha\beta}$ is the pull-back of $\eta_{MN}$. The parameter in front of the integral is the string tension, making the action dimensionless, with the \names{Regge} slope $\alpha'$ being related to the string length $l_{\rm s}$ by $\alpha'=l_{\rm s}^{2}$. We also defined $\dd^{2}\sigma = \dd\sigma^{0}\dd\sigma^{1}$. Due to the square root this action is not very suitable for a path integral quantisation. To get rid of the square root one can introduce an auxiliary metric\footnote{With the dependence of $\sigma$ we hereby actually mean a short form for $\sigma=(\sigma^{0},\sigma^{1})$ and by no means a single dependence on the string length only.} $h_{\alpha\beta}(\sigma)$ with $h_{\alpha\beta}h^{\beta\rho}=\delta_{\alpha}^{\rho}$ and define the \names{Polyakov} action
%
%
\begin{equation}
S_{\rm P} = -\frac{1}{4\pi\alpha'} \int\limits_{\Sigma}\dd^{2}\sigma\, \sqrt{-h}h^{\alpha\beta}\gamma_{\alpha\beta}.
\end{equation}
%
%
This action is equivalent to the \names{Nambu-Goto} action at the classical level, as can be proved by deriving the equations of motion for $h_{\alpha\beta}$, $\delta S/ \delta h^{\alpha\beta} = 0$. Defining the corresponding energy-momentum tensor $T_{\alpha\beta}$, such equations are the statement that $T_{\alpha\beta}$ has to vanish
%
%
\begin{equation}
T_{\alpha\beta} \equiv -4\pi\alpha' \frac{1}{\sqrt{-h}} \frac{\delta S}{\delta h^{\alpha\beta}} = \gamma_{\alpha\beta} - \frac{1}{2} h_{\alpha\beta} h^{\rho\sigma}\gamma_{\rho\sigma} = 0.
\end{equation}
%
%
From this follows
%
%
\begin{equation}
\sqrt{-\gamma} = \frac{1}{2} \sqrt{-h} h^{\rho\sigma} \gamma_{\rho\sigma},
\end{equation}
%
%
and thus the classical equivalence of the actions $S_{\rm NG}$ and $S_{\rm P}$. The symmetries of $S_{\rm P}$ are
%
%
\begin{itemize}
\item\textit{Global D-dimensional Poincarè invariance}
%
\begin{equation}
X^{M} \rightarrow \widetilde{X}^{M} = \Lambda_{N}^{M} X^{N} + a^{M}, \qquad \delta	h_{\alpha\beta} = 0,
\end{equation}
%
with $\Lambda_{N}^{M}$ and $a^{M}$ being $D$-dimensional \names{Lorentz} transformations and spacetime translations.
%
%
\item\textit{Reparametrisation (or diffeomorphism) invariance of the worldsheet}\\[2mm]
We can choose a reparametrisation of the worldsheet coordinates $\sigma^{\alpha}\rightarrow \tilde{\sigma}^{\alpha}(\sigma)$, where the fields $X^{M}$ and the 2d metric transform according
%
\begin{equation}
\begin{alignedat}{3}
X^{M}(\sigma) &\rightarrow \widetilde{X}^{M}(\tilde{\sigma}) = X^{M}(\sigma), \\
h_{\alpha\beta}(\sigma) &\rightarrow \tilde{h}_{\alpha\beta}(\tilde{\sigma}) = \frac{\del \sigma^{\alpha}}{\del \tilde{\sigma}^{\gamma}} \frac{\del \sigma^{\beta}}{\del \tilde{\sigma}^{\delta}}h_{\gamma\delta}(\sigma).
\end{alignedat}
\end{equation}
%
This is a gauge symmetry on the worldsheet.
%
%
\item\textit{Weyl invariance}\\
The action is invariant under the rescaling
\begin{align}
\widetilde{X}^{M}(\sigma) &= X^{M}(\sigma), &  \tilde{h}_{\alpha\beta}(\sigma)&=\mathit{\Omega}^{2}(\sigma)h_{\alpha\beta}(\sigma).
\end{align}
\end{itemize}
%
%
We also have to employ suitable boundary conditions. There are in fact two types of strings with different boundary conditions:
%
%
\begin{itemize}
\item \textit{Open strings}\\[2mm]
In this case we can set $\sigma_{0}=\pi$ and thus $\sigma\in (0,\pi)$. These type of strings satisfy either \names{Neumann}  boundary conditions
%
\begin{equation}
\del_{\sigma}X^{M}(\tau,\sigma)\big\vert_{\sigma=0,l} = 0
\end{equation}
%
or \names{Dirichlet} boundary conditions
%
\begin{equation}
\delta X^{M}(\tau,\sigma)\big\vert_{\sigma=0,l}=0.
\end{equation}
%
%
\item \textit{Closed strings}\\[2mm]
Here we usually set $\sigma_{0}=2\pi$, leading to $\sigma\in [0,2\pi)$. This type of string satisfies periodic boundary conditions
%
\begin{gather}
X^{M}(\tau,0)=X^{M}(\tau,2\pi), \qquad \del_{\sigma}X^{M}(\tau	,\sigma)\big\vert_{\sigma=0} = \del_{\sigma}X^{M}(\tau	,\sigma)\big\vert_{\sigma=2\pi}, \notag\\ h_{\alpha\beta}(\tau,0)=h_{\alpha\beta}(\tau,2\pi).
\end{gather}
\end{itemize}
%
%
One can then exploit the symmetries of the \names{Polyakov} action to give the string equations of motion a rather simple form. For instance, one can use the local symmetries to choose a convenient gauge in which the worldsheet metric is conformally flat
%
%
\begin{equation}
h_{\alpha\beta} = \mathit{\Omega}^{2}(\sigma) \eta_{\alpha\beta}, \quad \text{with} \quad \eta_{\alpha\beta} = \text{diag}(-,+).
\end{equation}
%
%
In this conformal gauge the \names{Polyakov} action takes the form
%
%
\begin{equation}
S_{\rm P} = \frac{1}{4\pi\alpha'} \int \dd^{2}\sigma\, \left(\del_{\tau}X^{M}\del_{\tau}X^{N}-\del_{\sigma}X^{M}\del_{\sigma}X^{N}\right)\eta_{MN}
\end{equation}
%
%
which is leading to the simple equations of motion
%
%
\begin{equation}
\left(\del_{\tau}^{2}-\del_{\sigma}^{2}\right)X^{M}=0.
\end{equation}
%
%
Their solutions are well known and can be derived by decomposition into \names{Fourier} modes. If one analyses the mass spectrum of the free string, there will appear states of negative mass square
%
%
\begin{equation}
\alpha' M^{2} = -1.
\end{equation}
%
%
Such states are called tachyons. They are not very well understood and give rise to instabilities in the theory that need to be removed.
%
%
%
%
%
%  - - - - - - - - -   superstring theory  - - - - - - - - - - -
%
%
%
%
%
\section{Superstring theory}
The theory considered so far describes only bosons and also gives rise to unphysical tachyon states. To make predictions on the real world which also contains non-bosonic particles like quarks and leptons, we require string theory to include fermions as well. It turns out that for the presence of fermions, string theory requires supersymmetry, a symmetry that introduces a fermionic superpartner to every bosonic field. The resulting theories are then called superstring theories. Among others there are two basic approaches to incorporate supersymmetry into string theory \cite{Becker:2007zj} which are both equivalent
%
%
\begin{itemize}
\item \textit{The Ramond-Neveu-Schwarz formalism} is supersymmetric on the worldsheet.
%
\item \textit{The Green-Schwarz (GS) fomalism} is supersymmetric in the target space.
\end{itemize}
%
%
Depending on how one applies supersymmetry there arise different possibilities on what kinds of strings the model will describe. With regard to the $AdS/CFT$ correspondence (see \autoref{sec: preleminaries}) we want to focus on Type IIB superstring theory which requires $D=10$ spacetime dimensions. Type II strings have both left and right moving fermions and the resulting spacetime theory has $\mathcal{N}=2$ supersymmetry, whereas B refers to the property of additional background gauge fields. In the following we choose the GS formalism to apply supersymmetry in the target space. \\
Usually we map the worldsheet into spacetime, but for a supersymmetric generalisation in the GS formalism we consider mapping the worldsheet into superspace, where the basic target space fields are
%
%
\begin{equation}
X^{M}(\tau,\sigma) \qquad \text{and} \qquad \mathit{\Theta}^{Aa}(\tau,\sigma).
\end{equation}
%
%
Here $\mathit{\Theta}^{Aa}$ are \names{Majorana-Weyl} (MW) spinors. The index $A$ is running from 1 to the number of supersymmetries $\mathcal{N}$ which is $\mathcal{N}=2$ for the Type II string theories, whereas $a$ labels the components of the spacetime spinor in $D$ dimensions with $a=1,\ldots,2^{D/2}$ if $D$ is even. So in the 10 dimensional case of Type IIB string theory we have two anti-commuting spinors
%
%
\begin{equation}
\mathit{\Theta}^{1a}, \mathit{\Theta}^{2a}\qquad \text{with} \qquad a=1,\ldots,32.
\end{equation}
%
%
Due to the MW property there are only 16 independent real components for each spinor. To write a \names{Polyakov}-like supersymmetric action in  \names{Minkowski} space one can start by 
%
%
\begin{equation}
S_{1} = -\frac{1}{4\pi\alpha'} \int \dd^{2}\sigma \; \sqrt{h}h^{\alpha\beta} \Pi^{M}_{\alpha}\Pi^{N}_{\beta} \eta_{MN},
\end{equation}
%
%
with the superfields
%
%
\begin{equation}
\Pi^{M}_{\alpha} \equiv \del_{\alpha}X^{M} - \bar{\mathit{\Theta}\;}\!\!^{A}\Gamma^{M}\del_{\alpha}\mathit{\Theta}^{A}
\label{eq: Pi_susy}
\end{equation}
%
%
and $\Gamma^{M}$ are 10d \names{Dirac} matrices and $\bar{\mathit{\Theta}\;}\!\!=\mathit{\Theta}^{\dagger}\Gamma_{0}$. By studying this kind of action for a point particle one finds that certain peculiarities arise when the particle is massless \cite{Becker:2007zj}. To circumvent this a \names{Wess-Zumino} (WZ) term $S_{2}$ is added which gives rise to a new symmetry in the full action $S=S_{1}+S_{2}$.  This so called $\kappa$-symmetry is a symmetry of the spinor fields
%
%
\begin{align}
\delta\bar{\mathit{\Theta}\;}\!\!^{1} &= \bar{\kappa}^{1}P_{-}, & \delta\bar{\mathit{\Theta}\;}\!\!^{2} &= \bar{\kappa}^{2}P_{+},
\end{align}
%
%
where $\kappa^{1}$ and $\kappa^{2}$ are arbitrary MW spinors with appropriate chirality and $P_{\pm}$ are projection operators. This symmetry acts like an internal gauge symmetry for the fermionic spinor fields. By fixing $\kappa$-symmetry half of the components of the fermionic variables decouple from the theory which means we are left with 8 independent real components for each spinor. A natural and convenient gauge choice is 
%
%
\begin{equation}
\Gamma^{+}\mathit{\Theta}^{A}=0, \qquad \text{where} \qquad \Gamma^{+}=\frac{1}{\sqrt{2}}\left( \Gamma^{0}+\Gamma^{9}\right).
\end{equation}
%
%
%
%
% - - - - - - - - -   strings on curved backgrounds  - - - - - - - - -- - - - -
%
%
%
%
\section{Strings in curved backgrounds}
For now we have only considered strings propagating in flat \names{Minkowski} target space. To give string theory a more general application, one also has to take other backgrounds into account. Therefore we promote the target space metric $\eta_{MN}\rightarrow g_{MN}(X)$, leading to the bosonic \names{Polyakov} action
%
%
\begin{equation}
S_{\rm P} = -\frac{1}{4\pi\alpha'}  \int\limits_{\Sigma} \dd^{2}\sigma\, \sqrt{-h}h^{\alpha\beta}g_{MN}(X) \del_{\alpha}X^{M}\del_{\beta}X^{N}.
\label{eq: S_P_nlsigma}
\end{equation}
%
%
This is called the \textit{bosonic non-linear string sigma model}. Hereby, the process of incorporating supersymmetry is not so straightforward anymore. Next to the flat 10d \names{Minkowski} spacetime there is also another maximally supersymmetric solution as a background for the Type IIB case which is the product $\AdSS$ of a five-dimensional anti-\names{de Sitter} space $AdS_{5}$ and a five-sphere $S^{5}$. This solution is supported by a self-dual \names{Ramond-Ramond} five-form flux. This is due to the presence of D-branes which can be seen as \names{Dirichlet}-like boundary conditions or a higher-dimensional generalisation of strings. In this context an open string with \names{Dirichlet} boundary conditions can be interpreted as an open string, ending on a hypersurface, called D-brane. Some of these D-branes can carry a conserved charge that ensures their stability. For an open string, ending on such charged branes, the spectrum is tachyon-free. In the following we consider three dimensional D3-branes which are required to formulate the $AdS/CFT$ correspondence. In the language of forms we can compute the electric and magnetic\footnote{Here we assume the existence of magnetic monopoles.} charges $\mu_{e}$ and $\mu_{m}$, using \names{Gauss's} law for D$p$-branes\footnote{D$p$-branes refer to D-branes of dimension $p$.} in a generalised formulation of \names{Maxwell's} equations
%
%
\begin{align}
\mu_{e} &= \int\limits_{S^{D-p-2}} \star F_{p+2},  &   \mu_{m} &= \int\limits_{S^{p+2}}  F_{p+2},
\end{align}
%
%
where $F_{p+2}$ is the field strength defined by
%
%
\begin{equation}
F_{n} = \frac{1}{n!} F_{\mu_{1}\cdots \mu_{n}} \dd x^{\mu_{1}} \wedge \cdots \wedge \dd x^{\mu_{n}},
\end{equation}
%
%
and $\star F_{p+2}$ is its \names{Hodge} dual. For our case with D3-branes with $p=3$ we find $F_{p+2}=F_{5}$ and in $D=10$ dimensions that $F_{5}=\star F_{5}$ is self-dual. Therefore both charges are computed via the flux supported by a self-dual five form through the sphere $S^{5}$. To construct an action one has to incorporate the field strength to the kinetic part of the action 
%
%
\begin{equation}
S_{1} = -\frac{1}{4\pi \alpha'} \int \dd^{2}\sigma\; \sqrt{-h}h^{\alpha\beta} G_{\alpha\beta},
\end{equation}
%
%
where $G_{\alpha\beta}$ needs to contain contracted terms of $\Pi^{\mu}_{\alpha}$ in (\ref{eq: Pi_susy}) that respects the appropriate background and also a field strength $\mathcal{F}_{\alpha\beta} = F_{\alpha\beta}+b_{\alpha\beta}$, where $F_{\alpha\beta}$ represents the five-form field strength manifested on the worldsheet and $b_{\alpha\beta}$ is a term that makes it supersymmetric. To achieve $\kappa$-symmetry one needs to add a WZ term which is the integral of a three-form $\Omega_{3}$ over a three-space $M_{3}$ that has the worldsheet as its boundary
%
%
\begin{equation}
S_{2} = \int\limits_{M_{3}} \Omega_{3}.
\end{equation}
%
%
Remarkably this problem has been solved by Metsaev and Tseytlin in \cite{Metsaev:1998it}, but we will not present the full procedure here due to its complexity and requirement of sophisticated analytical tools. We therefore refer the reader to the original publication. What we will have to deal with in the following is a special solution to a gauge fixed version of this $\AdSS$ GS string developed in \cite{Metsaev:1998it}. But before we come to that, we need to study the $AdS/CFT$ correspondence and derive a possible extraction of our observable of interest - the cusp anomaly of $\mathcal{N}=4$ SYM. 
