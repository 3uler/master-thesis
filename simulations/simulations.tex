\chapter{Simulations and observables}
In this chapter we will present some basic information on the implementation of the previously discussed algorithm and state the utilized parameters of the conducted simulations. Furthermore we need to discuss how to take the continuum limit of the observables of interest.
%
%
%
% - - - - - - -   implementation and sim param   - - - - - - - -
%
%
%
\section{Implementation and the continuum limit}
The executables for the Monte Carlo simulation are built from an implementation in the \names{Fortran 95} standard and compiled with an $\text{Intel}^{\circledR}$ \texttt{ifort} compiler\footnote{From the $\text{Intel}^{\circledR}$ Parallel Studio XE 2015.}. For the lattice we deploy several sizes for the spatial extent $L=8,10,12,16,24,32$, whereas the temporal extent is always twice of the spatial one $T=2L$ for an improved accuracy of the bosonic correlators. Thus we have a world volume of $V_{2}=2a^{2}L^{2}$. In the continuum model there are two "bare" parameters that determine its behaviour, the dimensionless coupling $g=\sqrt{\lambda}/4\pi$ and the mass scale $m$. To take the continuum limit it is necessary to set a line of constant physics when $a \to 0$ which is said to be the squared physical mass of the field excitations rescaled with the world volume
%
%
\begin{equation}
V_{2}m_{x}^{2} = \text{const}.
\label{eq: line_of_c_phys1}
\end{equation}
%
%
From a dimensional regularization scheme it is possible to find the corrections to the masses of the bosonic fields $x,x^{*}$ which read \cite{Giombi:2010bj}
%
%
\begin{equation}
m_{x}^{2}(g) = \frac{m^{2}}{2}\left(1 - \frac{1}{8g} + \mathcal{O}(g^{-2}) \right).
\label{eq: m_x}
\end{equation}
%
%
Together with (\ref{eq: line_of_c_phys1}) this leads to 
%
%
\begin{equation}
\frac{V_{2}m^{2}}{2} = (LM)^{2} = \text{const},
\label{eq: line_of_c_phys2}
\end{equation}
%
%
where $M=ma$ is the dimensionless lattice mass scale. The equality (\ref{eq: line_of_c_phys2}) relies on the hypothesis that $g$ is not (infinitely) renormalized. Further the validity of (\ref{eq: m_x}) in the discretized model needs to be verified in order the physical masses undergo only a finite renormalization. This claim is supported by studying $x,x^{*}$ correlators where indeed no presence $(1/a)$ divergences in the $m_{x}^{2}/m^{2}$ ratios can be found. Also for the large $g$ region they reach their expected continuum value of $1/2$. Having this in mind and also the result of the perturbative 1-loop free energy (\ref{eq: 1_loop}), we assume no further presence of a scale in the discretized model other than the lattice spacing $a$. Therefore any expectation value of an observable $\langle F_{\rm LAT}\rangle$ is a function of the "bare" input parameters $g,L$ and $M$
%
%
\begin{equation}
\langle F_{\rm LAT}\rangle = \langle F_{\rm LAT}(g,L,M)\rangle = \langle F(g) \rangle + \mathcal{O}(L^{-1}) + \mathcal{O}\left( e^{-LM}\right).
\end{equation}
%
%
For fixed $g$ one chooses a fixed $LM$, large enough to keep finite volume effects $\mathcal{O}\left( e^{-LM}\right)$ small. For each finite value $L$ there will be a difference of $\langle F_{\rm LAT}\rangle$ and its continuum equivalent by means of lattice artefacts $\mathcal{O}(L^{-1})$. The continuum limit $\langle F(g) \rangle$ is obtained via an extrapolation to infinite $L$.
%
%
%
%
%
%  - - - - - - - -  simulation parameters   - - - - - - - - - - - 
%
%
%
%
\section{Simulation parameters}
As mentioned we employ different lattice sizes varying between $L=8$ and $L=32$. In the HMC simulation the MD equations of motion are evaluated along a fictitious MC time $\tau$. For one trajectory we utilize a MC time of $\tau = 0.5$ with by default 100 integrator steps and therefore $\epsilon=\delta\tau = 0.005$. For the fractional powers of the fermion matrix we use a rational approximation (\ref{eq: rat_approx}) of degree $P=15$ with the two sets of parameters stated in \autoref{tab: rat_app_coef}. We checked for a subset of the configurations that is accuracy is always better than $10^{-3}$ for $\xi^{\dagger}\big(\mathcal{O}_{\rm F}^{\dagger}\mathcal{O}_{\rm F}\big)^{-\frac{1}{4}}\xi$.
%
%
\begin{table}[h]
\begin{tabular}{ccll}
\toprule
$\rho$ &$i$ &  \hspace{2cm}$\alpha_{i}$ &  \hspace{2cm}$\beta_{i}$ \\ 
\midrule
  & 0 & $\hspace{10pt} 3.2148873149863206$ & \hspace{2cm}- \\ 
  & 1 & $-2.2977600408751347\cdot 10^{-9}$ & $5.5367335615411457\cdot 10^{-8}$ \\ 
  & 2 & $-1.6898103706901084\cdot 10^{-8}$ & $4.6910257304582898\cdot 10^{-7}$ \\ 
  & 3 & $-1.1099658368596436\cdot 10^{-7}$ & $2.6768223190551614\cdot 10^{-6}$ \\ 
  & 4 & $-7.2162146587729939\cdot 10^{-7}$ & $1.4319657256375662\cdot 10^{-5}$ \\ 
  & 5 & $-4.6841070484595924\cdot 10^{-6}$ & $7.5694473187855338\cdot 10^{-5}$ \\ 
  & 6 & $-3.0396303865820389\cdot 10^{-5}$ & $3.9922490005559548\cdot 10^{-4}$ \\ 
1/8  & 7 & $-1.9723870959636086\cdot 10^{-4}$ & $2.1046795395127538\cdot 10^{-3}$ \\ 
  & 8 & $-1.2798599250624023\cdot 10^{-3}$ & $1.1094832053548640\cdot 10^{-2}$ \\ 
  & 9 & $-8.3051856063983548\cdot 10^{-3}$ & $5.8486687698920667\cdot 10^{-2}$ \\ 
  & 10 & $-5.3904877281192094\cdot 10^{-2}$ & $3.0834388405073770\cdot 10^{-1}$ \\ 
  & 11 & $-3.5026088217184553\cdot 10^{-1}$ & $1.6264534005778293$ \\ 
  & 12 & $-2.2893521967679966$ & $8.6030459456576764$ \\ 
  & 13 & $-1.5436668340425719\cdot 10 $& $4.6179583183155444\cdot 10$ \\ 
  & 14 & $-1.2297861076048798\cdot 10^{2}$ & $2.6854965277696181\cdot 10^{2}$ \\ 
  & 15 & $-2.6252652966414048\cdot 10^{3} $& $2.6004158696112045\cdot 10^{3} $\\ 
%\bottomrule
%\end{tabular} 
%\end{table}
%\vspace{0.5cm}
%\begin{table}[ht!]
%\begin{tabular}{ccll}
%\toprule
%$\rho=-\frac{1}{4}$ &$i$ &  \hspace{2cm}$\alpha_{i}$ &  \hspace{2cm}$\beta_{i}$ \\ 
\midrule
  & 0 & $9.5797060554725838\cdot 10^{-2}$ & \hspace{2cm}- \\ 
  & 1 & $1.7701746700099842\cdot 10^{-6}$ & $3.1085594175442315\cdot 10^{-8}$ \\ 
  & 2 & $5.8705983656937455\cdot 10^{-6}$ & $3.2994455960441383\cdot 10^{-7}$ \\ 
  & 3 & $1.9961158693570120\cdot 10^{-5}$ & $1.9424842756552213\cdot 10^{-6}$ \\ 
  & 4 & $6.9125367600088173\cdot 10^{-5}$ & $1.0453359626231250\cdot 10^{-5}$ \\ 
  & 5 & $2.4032965323696816\cdot 10^{-4}$ & $5.5337819905761986\cdot 10^{-5}$ \\ 
  & 6 & $8.3620125835371663\cdot 10^{-4}$ & $2.9204178440857227\cdot 10^{-4}$ \\ 
-1/4  & 7 & $2.9099006745502945\cdot 10^{-3}$  & $1.5403300046437174\cdot 10^{-3}$ \\ 
  & 8 & $1.0126504714418652\cdot 10^{-2}$ & $8.1233558140562465\cdot 10^{-3}$ \\ 
  & 9 & $3.5241454044660878\cdot 10^{-2}$ & $4.2840454273820550\cdot 10^{-2}$ \\ 
  & 10 & $1.2266034741624667\cdot 10^{-1}$ & $2.2594500626442715\cdot 10^{-1}$ \\ 
  & 11 & $4.2721681852328125\cdot 10^{-1}$ & $1.1921171782283737$ \\ 
  & 12 & 1.4932820692676758 & 6.3026182343759860 \\ 
  & 13 & 5.3188766358452595 & $3.3683411978650057\cdot 10$ \\ 
  & 14 & $2.0944763089672641\cdot 10^{1}$ & $1.9083658214156412\cdot 10^{2}$ \\ 
  & 15 & $1.4525770103354523\cdot 10^{2}$ & $1.5386784635765257\cdot 10^{3}$ \\ 
\bottomrule
\end{tabular} 
\caption{Two sets of coefficients used for the rational approximation (\ref{eq: rat_approx}) sufficient for the exponents $\rho=1/8$ and $\rho=-1/4$, respectively.\label{tab: rat_app_coef}}
\end{table}