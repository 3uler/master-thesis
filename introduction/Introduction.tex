\documentclass[12pt,a4paper,twoside]{book}
\usepackage[utf8]{inputenc}
\usepackage[german,english]{babel} 
\usepackage{amsmath}
\usepackage{amsfonts}
\usepackage{amssymb}
\usepackage{graphicx}
%\usepackage{subfig}
\usepackage[left=4cm,right=3.5cm,top=2.5cm,bottom=3.5cm]{geometry}
\usepackage{setspace}
\usepackage{babelbib}
\usepackage{bm}
\usepackage{upgreek}
\usepackage[nottoc]{tocbibind}
\usepackage{tikz}
\usepackage{dsfont}
%\usepackage{mathabx}
\usepackage[toc,page]{appendix}
%erweiterungen
\usepackage{titlesec,titletoc}	
%\usepackage{scrtime}
\usepackage{textcomp}
\usepackage{upgreek}
\usepackage{framed, color} 
\usepackage[colorlinks=false,hidelinks]{hyperref}
\usepackage{tcolorbox}
\usepackage{anyfontsize}

\usepackage{ifthen}

\usepackage[labeled,resetlabels]{multibib}	%%meherere literaturverzeichnisse

% declare new math alphabets for special bolt and italic fonts
\DeclareMathAlphabet{\mathcalbf}{OMS}{cmsy}{b}{n}
\DeclareMathAlphabet{\mathitbf}{OT1}{cmss}{bx}{it}

%change name displayed as appendix title
\renewcommand{\appendixpagename}{\sffamily \bfseries \huge Appendix}

\graphicspath{{/Users/Philipp/Documents/Uni/Masterarbeit/latex/draft/graphics/}}








%layout einstellungen

%equation nummerierungsformat
\renewcommand{\theequation}{\arabic{section}.\arabic{equation}} 
\numberwithin{equation}{section}


%caption-formatierung

\numberwithin{figure}{section}	%abbildungszähler
\numberwithin{table}{section}	%tabellenzähler

%sidecap formatierung (ergänzung zur figure und table umgebung)
\usepackage[outercaption]{sidecap}

\usepackage[margin=0pt,font=small,labelformat=simple,labelfont=bf,labelsep=space, justification=justified, hypcap]{caption}
%\captionsetup[figure]{name=Figure}
%\captionsetup[table]{name=Table}
%\captionsetup[table]{position=above}
%zur positionuierung einer SC caption
\makeatletter
\newenvironment{SCtopfig}{\SC@float[t]{figure}}{\endSC@float}
\makeatother






%header einstellungen
\usepackage{fancyhdr}
\pagestyle{fancy}
\fancyhead{}
\fancyhead[RO]{ \nouppercase{\rightmark}}
\fancyhead[LE]{ \nouppercase{\leftmark}} %\slshape

\fancyfoot{}
\fancyfoot[CE,CO]{\thepage}




%Einstellung serifenlose überschriften
\titleformat{\chapter}[hang]{\sffamily \bfseries \Large}{\framebox[0.9cm]{\mdseries\fontsize{40}{0}\selectfont\thechapter}}{ 0.5cm}{ }[\vspace{-0.65cm}\rule{\textwidth}{1pt}]
\titleformat*{\section}{\sffamily \bfseries \Large}
\titleformat*{\subsection}{\sffamily \bfseries \large}
\titleformat*{\subsubsection}{\sffamily \bfseries \normalsize}









%%EIGENDEFINITIONEN

\definecolor{grey}{rgb}{0.925,0.925,0.925}
\newcommand{\ueberschrift}[1]{\vspace{2mm}{\sffamily\normalsize\bfseries {#1} \vspace{2mm}}}
\newcommand{\ergebnis}[1]{\vspace{2mm}
\fcolorbox{black}{grey}{\parbox{\columnwidth}{
\begin{equation}
{#1}
\end{equation}
}}\vspace{2mm}}
\newcommand{\abs}[1]{\ensuremath{\left\vert#1\right\vert}}
%differential d
\newcommand{\dd}{\ensuremath{\text{d}}}
%differential del
\newcommand{\del}{\ensuremath{\partial}}
%index real part
\newcommand{\R}{\ensuremath{^{\text{R}}}}
\newcommand{\dR}{\ensuremath{_{\text{R}}}}
%index imag part
\newcommand{\I}{\ensuremath{^{\text{I}}}}
\newcommand{\dI}{\ensuremath{_{\text{I}}}}

% AdS_5 x S^5
\newcommand{\AdSS}{\ensuremath{AdS_{5}\times S^{5}}}



% definition of names to be written in a specific font
\newcommand{\names}[1]{\textsc{#1}}

% Pfaffian
\newcommand{\pf}{\ensuremath{\text{Pf}\,}}

%backward vector arrow
\makeatletter
\DeclareRobustCommand{\cev}[1]{%
  \mathpalette\do@cev{#1}%
}
\newcommand{\do@cev}[2]{%
  \fix@cev{#1}{+}%
  \reflectbox{$\m@th#1\vec{\reflectbox{$\fix@cev{#1}{-}\m@th#1#2\fix@cev{#1}{+}$}}$}%
  \fix@cev{#1}{-}%
}
\newcommand{\fix@cev}[2]{%
  \ifx#1\displaystyle
    \mkern#23mu
  \else
    \ifx#1\textstyle
      \mkern#23mu
    \else
      \ifx#1\scriptstyle
        \mkern#22mu
      \else
        \mkern#22mu
      \fi
    \fi
  \fi
}
\makeatother



\title{Geodäten im Gravitationsfeld geladener Staubwolken}
\author{Philipp Töpfer}
\begin{document}
\section{Introduction}
%
%
Towards our understanding of the basic principles of how elementary particles behave and why, humanity has come a long way. It has been a constant path of discovering and unifying, which has led us to the standard model, the state of the art theory to tell us which particles are there and to predict how they behave and interact. But one last important step of unification has not been made so far. Till now no one was able to unify the three basic forces included in the standard model with the fourth elementary force - gravity. Gravity, which is described through general relativity, is not renormalizable and since renormalization is an important technique to obtain reasonable solutions in the  standard model, an ad hoc unification of both theories via a quantum field theoretical approach fails. This is where string theory comes into mind. String theory includes spin two excitations, that match gravitons, which provides the possibility to consider gravity within string theory calculations and since string theory in its simplest form is a two dimensional quantum field theory (QFT), it is also renormalizable, which makes it an important candidate for a possible theory of everything. But there are various problems. For one there is not a single string theory, but a variety of them and each is a certain approximative limit of an underlying theory called M-theory, which is not known for sure. Also each of these theories predicts a high number of space-time dimensions, which have to be compactified to lead to our known four. There is a vast amount of possibilities how to perform these compaktification and each leads to different results. Therefore it has not been possible to make satisfying predictions about the reality that we live in with the help of string theory. \\

Nonetheless we want to gain a deeper understanding of the theory and how it could lead to more reasonable predictions. In 1997 Juan Maldacena proposed the $AdS/CFT$ correspondence, which relates string theories in anti-de Sitter space to certain conformal field theories \cite{maldacena1}, some of them similar in some ways to quantum chromodynamics (QCD), which makes this conjecture so interesting. Till now there is no rigorous proof that the conjecture persists in every case. So it is the aim of many researchers to investigate further to which extend the conjecture holds true. In perturbation theory one is limited to the cases of strong and low coupling, where one can make explicit calculations to check for the correctness of the proposal. Everything in between is not accessible to perturbation theory and therefore we want to take a different approach and make calculations via a lattice simulation.\\

The aim of this thesis is to study a special toy model where a type IIB string theory in $AdS_{5}\times S^{5}$ background is dual to a $\mathcal{N}=4$ super-Yang-Mills theory (SYM). Here we are interested in calculating the cusp anomalous dimension of a Wilson loop in $\mathcal{N}=4$ SYM by its dual minimal surface on the string theory side. This is possible due to the Maldacena conjecture, stating a duality of Wilson loops in $\mathcal{N}=4$ SYM and certain minimal surfaces in type IIB string theory on $AdS_{5}\times S^{5}$ background \cite{maldacena2}. On the gauge theory side several lattice calculations have been made by Catterall et al. \cite{Catterall_physrept, Bergner:2016sbv}. Doing this kind of calculations on the string theory side is rather new and has been first proposed by Roiban \cite{Roiban}. The aim of this thesis is therefore to 
%
%
\newpage
\bibliographystyle{nb}%{babunsrt} 
\bibliography{Master}
%
%
\end{document}