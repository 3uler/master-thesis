\chapter{Introduction}
Towards our understanding of the underlying principles of nature, humanity has come a long way, covering the realm of subatomic particles up to the astronomic scale of stars and galaxies and even the universe as a whole. It has been a constant path of discovering and unifying knowledge to see a bigger picture. This path has led us to \names{Einstein's} theory of general relativity (GR), using the idea to manifest gravity via the curvature of the unified framework of space and time, known as spacetime. Many predictions of this theory proved to be correct, like the recent detection of gravitational waves by the LIGO and virgo collaborations \cite{Abbott:2016blz,Abbott:2016nmj}. On the other hand there is quantum field theory (QFT) that emerged from a field theoretic generalization of relativistic quantum mechanics and was elaborated via quantum electrodynamics (QED) and quantum chromodynamics (QCD) to the standard model (SM) which is the state of the art theory of elementary particle physics that has been completed by the recent discovery of the \names{Higgs} boson \cite{Chatrchyan:2012xdj,Aad:2012tfa}. Nonetheless it was not possible to unify the three fundamental forces condensed in the standard model with the fourth elementary fore - gravity. The current mathematical treatment of QFT requires renormalization as a tool to extract finite physical observables. But this requirement also restricts a theory in its universality. Unfortunately GR is not renormalizable and therefore not ad hoc compatible to QFT.\\
A possible cure to this peculiarity might come from string theory which emerged from the study of hadrons and \names{Regge} trajectories \cite{Veneziano:1968yb}. Further discoveries of superstring states with spin two, indicating the presence of gravitons and therefore gravity, made string theory a promising candidate for a unified theory. But string theory requires additional dimensions and a high energy scale for a phenomenological investigation which are both requirements that the abilities of modern detectors and accelerators still fail to realize. Therefore one is left to study string theory mainly in a theoretical framework which has made major progress in the last decades. One of the most remarkable discoveries is certainly the conjecture of the anti-\names{de Sitter} / conformal field theory ($AdS/CFT$) correspondence made by Maldacena \cite{maldacena1}. It relates a maximally supersymmetric version of a non-\names{Abelian} gauge theory like QCD to a string theory and states their mathematical equivalence. This statement is remarkable in the sense that it relates a theory of quantum gravity to a gauge theory on a flat space that was thought to have no gravitational relation at all.\\
The aim of this thesis is to contribute to a deeper understanding of the $AdS/CFT$ corresponding in a numerical study on the fundamental case where the conjecture relates $\mathcal{N}=4\; SU(N)$ super \names{Yang-Mills} (SYM) theory in four dimensions to a Type IIB superstring theory on an $\AdSS$ background, supported by a \names{Ramond-Ramond} five-form flux.\\
As a non-\names{Abelian} gauge theory it is natural to approach $\mathcal{N}=4$ SYM with the same numerical techniques known from lattice QCD, where an interesting program has been carried out by \names{Catterall} et. al. \cite{Catterall_physrept, Bergner:2016sbv}. Alternatively, one could discretize the worldsheet spanned by the \names{Green-Schwarz} string embedded in $\AdSS$. This route has been first explored in \cite{Roiban} where the observable of major concern - the cusp anomaly of $\mathcal{N}=4$ SYM - has been investigated with lattice methods from a string theory point of view. The thesis at hand is mainly based on a proceeding of this lattice approach to the \names{Green-Schwarz} string  proposed in \cite{Bianchi:2016cyv,Forini:2016sot}.