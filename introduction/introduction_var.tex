\chapter{Introduction}
Towards our understanding of the underlying principles of nature, humanity has come a long way, covering the realm of subatomic particles up to the astronomic scale of stars and galaxies and even the universe as a whole. It has been a constant path of discovering and unifying knowledge to see a bigger picture. This path has led us to \names{Einstein's} theory of general relativity (GR), using the idea to manifest gravity via the curvature of the combined framework of space and time, known as spacetime. Many predictions of this theory proved to be correct, like the recent detection of gravitational waves by the LIGO and virgo collaborations \cite{Abbott:2016blz,Abbott:2016nmj}. On the other hand there is quantum field theory (QFT) that emerged from a field theoretic generalisation of relativistic quantum mechanics and was elaborated via quantum electrodynamics (QED) and quantum chromodynamics (QCD) to the standard model (SM) which is the state of the art theory of elementary particle physics that has been completed by the recent discovery of the \names{Higgs} boson \cite{Chatrchyan:2012xdj,Aad:2012tfa}. Nonetheless it was not possible to unify the three fundamental forces condensed in the standard model with the fourth elementary force - gravity. The current mathematical treatment of QFT requires renormalisation as a tool to extract finite physical observables. But this requirement also restricts a theory in its universality. Unfortunately GR is not renormalisable and therefore not ad hoc compatible to QFT.\\
A possible cure to this peculiarity might come from string theory which emerged from the study of hadrons and \names{Regge} trajectories \cite{Veneziano:1968yb}. Further discoveries of superstring states with spin two, indicating the presence of gravitons and therefore gravity, made string theory a promising candidate for a unified theory. But string theory requires additional dimensions and a high energy scale for a phenomenological investigation which are both requirements that the abilities of modern detectors and accelerators still fail to realise. Therefore one is left to study string theory mainly in a theoretical framework which has made major progress in the last decades. One of the most remarkable discoveries is certainly the conjecture of the anti-\names{de Sitter} / conformal field theory ($AdS/CFT$) correspondence made by \names{Maldacena} \cite{maldacena1}. It relates a maximally supersymmetric version of a non-\names{Abelian} gauge theory like QCD to a string theory and states their mathematical equivalence. This statement is remarkable in the sense that it relates a theory of quantum gravity to a gauge theory on a flat space that was thought to have no gravitational relation at all.\\
The aim of this thesis is to contribute to a deeper understanding of the $AdS/CFT$ corresponding in a numerical study on the fundamental case where the conjecture relates $\mathcal{N}=4\; SU(N)$ super \names{Yang-Mills} (SYM) theory in four dimensions to a Type IIB superstring theory on an $\AdSS$ background, supported by a \names{Ramond-Ramond} five-form flux.\\
As a non-\names{Abelian} gauge theory it is natural to approach $\mathcal{N}=4$ SYM with the same numerical techniques known from lattice QCD, where an interesting program has been carried out by \names{Catterall} et. al. \cite{Catterall_physrept, Bergner:2016sbv}. Alternatively, one could discretise the worldsheet spanned by the \names{Green-Schwarz} string embedded in $\AdSS$. This route has been first explored in \cite{Roiban} where the observable of major concern - the cusp anomaly of $\mathcal{N}=4$ SYM - has been investigated with lattice methods from a string theory point of view. The thesis at hand is mainly based on a proceeding of this lattice approach to the \names{Green-Schwarz} string  proposed in \cite{Bianchi:2016cyv,Forini:2016sot}. \\[0.5cm]
%
%
In $\mathcal{N}=4$ SYM the renormalised vacuum expectation value (vev) of a \names{Wilson} loop along a light-like cusp is governed by the cusp anomaly (or ``scaling function") $f(g)$, a function of the coupling $g=\sqrt{\lambda}/4\pi$, where $\lambda$ is the \names{'t Hooft} coupling of the $AdS/CFT$ dual gauge theory. The \names{Maldacena} conjecture states an exact equivalence of the vev of any \names{Wilson} loop with the path integral of an open string ending on the boundary of $AdS_{5}$, where the 4d gauge theory lives \cite{maldacena2}. In the case of a cusped contour $\mathcal{C}_{\rm cusp}$ for the \names{Wilson} loop we find
%
%
\begin{equation}
\langle \mathcal{W}[\mathcal{C}_{\rm cusp}] \rangle \equiv Z_{\rm cusp} = \int \mathcal{D}\delta X\,\mathcal{D}\delta\mathit{\Psi} \; e^{-S_{\rm cusp}[X_{\rm cl}+\delta X, \delta\mathit{\Psi} ]  } = e^{-\frac{1}{8}f(g) V_{2}},
\label{eq: intro_W_cusp}
\end{equation}
%
%
where $S_{\rm cusp}$ is the fluctuation action around a vacuum with the classical solution $X_{\rm cl}$, describing a string worldsheet of an open string that ends on a null cusp \cite{Giombi:2009gd}. The fields $X(s,t),\;\mathit{\Psi}(s,t)$ are the bosonic and fermionic string coordinates, depending on the spatial and temporal worldsheet coordinates $s$ and $t$, respectively. The factor $V_{2}=\int \dd s \dd t$ is the worldsheet volume of the string. Approximations of the scaling function $f(g)$ can be obtained from the perturbative limits of either gauge theory \cite{Bern:2006ew} $(g \ll 1)$ or string theory via sigma-model loop expansion \cite{Giombi:2009gd,Gubser:2002tv,Frolov:2002av} $(g \gg 1)$. With help of a thermodynamic \names{Bethe} Ansatz it is also possible to derive an integral equation \cite{Beisert:2006ez} that determines $f(g)$ exactly for any value of the coupling.\\[0.5cm]
%
%
In lattice field theory it is common to investigate vevs of certain observables of interest. To study the scaling function in a lattice approach we need to find an observable vev that that is related to $f(g)$. Such an observable is given by the ``cusp" action
%
%
\begin{equation}
\langle S_{\rm cusp} \rangle = \frac{\int \mathcal{D}\delta X\,\mathcal{D}\delta\mathit{\Psi} \; S_{\rm cusp} \;e^{-S_{\rm cusp}}}{\int \mathcal{D}\delta X\,\mathcal{D}\delta\mathit{\Psi} \; e^{-S_{\rm cusp}}} = -g \frac{\dd \ln Z_{\rm cusp}}{\dd g} \equiv g\frac{V_{2}}{2} f'(g),
\end{equation}
%
%
where we are supposed to obtain information on the derivative of the scaling function.\\
This approach to a numerical treatment of the \names{Green-Schwarz} string might seem a bit restricted in its abilities at a first glance, due to the limitation to a certain gauge fixed vacuum. But this attempt is only a first step towards a new area of lattice field theory that might open a wide range of possibilities. The model at hand is despite a highly non-trivial but two-dimensional one that only involves scalar fields which makes it computationally cheaper and more economical in the consumption of memory. In general, one merit of the analysis initiated in \cite{Roiban} and that is readdressed in \cite{Bianchi:2016cyv,Forini:2016sot} and here is to explore another route via which lattice simulations could become a potentially efficient tool in numerical holography.\\[0.5cm]
%
%
The thesis proceeds in the following way.\\
%
Chapter \ref{ch: string_basics} gives a brief introduction of superstring theory and the non-linear sigma-model on curved backgrounds.\\
%
In \autoref{sec: preleminaries} we present the basic concepts of the $AdS/CFT$ correspondence and other theoretical preliminaries to motivate the equivalence stated in (\ref{eq: intro_W_cusp}). The chapter finishes with the derivation of the ``cusp" fluctuation action motivated from \cite{Metsaev:2000yu,Metsaev:2000yf,Giombi:2009gd}.\\
%
Chapter \ref{ch: lattice_basics} forges a bridge between the continuous model and its discretised version by providing the essential tools on discretizing fields and operators and addresses a suitable algorithm to perform simulations.\\
%
In \autoref{sec: towards_lat} we use the previously presented methods to construct a lattice version of the action derived in \autoref{sec: preleminaries}. We study the properties of the resulting fermion matrix and present the full rational hybrid Monte Carlo algorithm that is used for the simulation.\\
%
Chapter \ref{ch: parameters} gives some technical details and lists of simulation parameters. We also present the observables of interest and how to conduct a continuum extrapolation from the observations on the lattice.\\
%
Chapter 7 illustrates the simulation results and the thesis finishes with a conclusion and an outlook on further research possibilities in chapter 8.\\
%
Details on several topics can be found in the appendices. We encourage the reader to consult them in parallel while reading the reading the main text.
