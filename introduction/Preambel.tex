\documentclass[12pt,a4paper,twoside]{report}
\usepackage[utf8]{inputenc}
\usepackage[german,english]{babel} 
\usepackage{amsmath}
\usepackage{amsfonts}
\usepackage{amssymb}
\usepackage{graphicx}
%\usepackage{subfig}
\usepackage[left=2.5cm,right=2.5cm,top=2.5cm,bottom=2cm]{geometry}
\usepackage{setspace}
\usepackage{babelbib}
\usepackage{bm}
\usepackage{upgreek}
\usepackage[nottoc]{tocbibind}


%erweiterungen
\usepackage{titlesec,titletoc}	
%\usepackage{scrtime}
\usepackage{textcomp}
\usepackage{upgreek}
\usepackage{framed, color} 
\usepackage[colorlinks=false,hidelinks]{hyperref}

\usepackage{ifthen}

\usepackage[labeled,resetlabels]{multibib}	%%meherere literaturverzeichnisse














%layout einstellungen

%equation nummerierungsformat
\renewcommand{\theequation}{\arabic{section}.\arabic{equation}} 
\numberwithin{equation}{section}


%caption-formatierung

\numberwithin{figure}{section}	%abbildungszähler
\numberwithin{table}{section}	%tabellenzähler

%sidecap formatierung (ergänzung zur figure und table umgebung)
\usepackage[outercaption]{sidecap}

\usepackage[margin=0pt,font=small,labelformat=simple,labelfont=bf,labelsep=space, justification=justified, hypcap]{caption}
\captionsetup[figure]{name=Abb.}
\captionsetup[table]{name=Tabelle}
\captionsetup[table]{position=above}
%zur positionuierung einer SC caption
\makeatletter
\newenvironment{SCtopfig}{\SC@float[t]{figure}}{\endSC@float}
\makeatother






%header einstellungen
\usepackage{fancyhdr}
\pagestyle{fancy}
\fancyhead{}
\fancyhead[RO]{\slshape \nouppercase{\rightmark}}
\fancyhead[LE]{\slshape \nouppercase{\leftmark}}

\fancyfoot{}
\fancyfoot[CE,CO]{\thepage}




%Einstellung serifenlose überschriften
\titleformat*{\section}{\sffamily \bfseries \Large}
\titleformat*{\subsection}{\sffamily \bfseries \large}
\titleformat*{\subsubsection}{\sffamily \bfseries \normalsize}









%%EIGENDEFINITIONEN

\definecolor{grey}{rgb}{0.925,0.925,0.925}
\newcommand{\ueberschrift}[1]{\vspace{2mm}{\sffamily\normalsize\bfseries {#1} \vspace{2mm}}}
\newcommand{\ergebnis}[1]{\vspace{2mm}
\fcolorbox{black}{grey}{\parbox{\columnwidth}{
\begin{equation}
{#1}
\end{equation}
}}\vspace{2mm}}
\newcommand{\abs}[1]{\ensuremath{\left\vert#1\right\vert}}

\title{Geodäten im Gravitationsfeld geladener Staubwolken}
\author{Philipp Töpfer}